\section{Methods} \label{sec:methods}

\subsection{Statistical Framework} \label{sec:statistical_framework}

We employ the typical hierarchical Bayesian inference framework to infer the properties of the population of merging compact binaries given a catalog of observations. The rate of compact binary mergers is modeled as an inhomogeneous Poisson point process \citep{10.1093/mnras/stz896}, with the merger rate per comoving volume $V_c$ \citep{astro-ph/9905116}, source-frame time $T_\text{src}$ and binary parameters $\theta$ defined as:

\begin{equation} \label{eq:rate}
    \frac{dN}{dV_cdt_\mathrm{src}d\theta} = \frac{dN}{dV_cdt_\mathrm{src}} p(\theta | \Lambda) = \mathcal{R} p(\theta | \Lambda)
\end{equation}

\noindent with $p(\theta | \Lambda)$ the population model, $\mathcal{R}$ the merger rate, and $\Lambda$ the set of population hyperparameters. Following other population studies \citep{10.1093/mnras/stz896,2021ApJ...913L...7A,2111.03634,2007.05579}, we use the hierarchical likelihood \citep{10.1063/1.1835214} that incorporates selection effects and marginalizes over the merger rate as: 

\begin{equation} \label{eq:likelihood}
    \mathcal{L}(\bm{d} | \Lambda) \propto \frac{1}{\xi(\Lambda)} \prod_{i=1}^{N_\mathrm{det}} \int d\theta \mathcal{L}(d_i | \theta) p(\theta | \Lambda)
\end{equation}

\noindent Above, $\bm{d}$ is the set of data containing $N_\mathrm{det}$ observed events, $\mathcal{L}(d_i | \theta)$ is the individual event likelihood function for the $i$th event given parameters $\theta$ and $\xi(\Lambda)$ is the fraction of merging binaries we expect to detect, given a population described by $\Lambda$. The integral of the individual event likelihoods marginalizes over the uncertainty in each event's binary parameter estimation, and is calculated with Monte Carlo integration and by importance sampling, reweighing each set of posterior samples to the likelihood. The detection fraction is calculated with:

\begin{equation} \label{eq:detfrac}
    \xi(\Lambda) = \int d\theta p_\mathrm{det}(\theta) p(\theta | \Lambda)
\end{equation}

\noindent with $p_\mathrm{det}(\theta)$ the probability of detecting a binary merger with parameters $\theta$. We calculate this fraction using simulated compact merger signals that were evaluated with the same search algorithms that produced the catalog of observations. With the signals that were successfully detected, we again use Monte Carlo integration to get the overall detection efficiency, $\xi(\Lambda)$ \citep{1712.00482, 1904.10879, 2204.00461}.



% \remove{To model multiple subpopulations, similar to how \citet{1302.5341} employed discrete latent categorical parameters to separate foreground from background, we introduce discrete latent variables to probabilistically assign events to categories with unique mass and spin distributions. Incorporating these discrete variables during inference allows us to easily infer each BBH's association with each category, in addition to the posterior distributions for astrophysical branching ratios (also referred to as mixing fractions). Previous applications of mixture models to the binary catalog \citep{10.3847/2041-8213/abe949,2022arXiv220902206T} implicitly marginalize over event categories. The population properties inferred using these approaches are identical, and event categorization probabilities can be calculated, but sampling these latent variables enables detailed investigation of correlations between categorization of events and population properties.}

% \remove{For $M$ subpopulations in a catalog of $N_\mathrm{det}$ detections, we add a latent variable $k_i$ for each merger that can be $M$ different discrete values between $0$ and $M-1$, each associated with a separate model, $p_{k_i}(\theta | \Lambda_{k_i})$, and hyperparameters, $\Lambda_{k_i}$. Evaluating the model (or hyper-prior) for the $i^\mathrm{th}$ event with binary parameters, $\theta_i$, given latent variable $k_i$ and hyperparameters $\Lambda_{k_i}$, we have:}

\begin{equation} \label{eq:latent}
    p(\theta_i | \Lambda, k_i) = p_{k_i}(\theta_i | \Lambda_{k_i})
\end{equation}

% \noindent \remove{To construct our probabilistic model, we first sample $p_{\lambda} \sim \mathcal{D}(M)$, from an M-dimensional Dirichlet distribution of equal weights, representing the astrophysical branching ratios $\lambda_{k_i}$ of each subpopulation. Then each of the $N_\mathrm{det}$ discrete latent variables are sampled from a categorical distribution with each category, $k_i$, having probability $p_{\lambda_{k_i}}$. Within the \textsc{NumPyro} \citep{1810.09538,1912.11554} probabilistic programming language, we use the implementation of \texttt{DiscreteHMCGibbs} \citep{Liu1996PeskunsTA} to sample the discrete latent variables, while using the \texttt{NUTS} \citep{1111.4246} sampler for continuous variables. While this approach may seem computationally expensive, we find that the conditional distributions over discrete latent variables enable Gibbs sampling with similar costs and speeds to the equivalent approach that marginalizes over each discrete latent variable, $k_i$. We find the same results with either approach and only slight performance differences that depend on specific model specifications, and thus opt for the approach without marginalization. This method also has the advantage that we get posterior distributions on each event's subpopulation assignment without extra steps.}

\subsection{Astrophysical Mixture Models} \label{sec:astromodels} 

Given the recent evidence for peaks in the BBH primary mass spectrum at $10 M_{\odot}$, $35 M_{\odot}$, and suggestions of a potential feature at $\sim20 M_{\odot}$ \citep{10.3847/2041-8213/aa9bf6, 10.3847/1538-4357/aab34c, 10.3847/2041-8213/ab3800, 2021ApJ...913L...7A, 2111.03634, 2022ApJ...928..155T,2022arXiv221012834E}, we chose mixture models similar to the \textsc{Multi Spin} model in \cite{2021ApJ...913L...7A}, which is characterized by a power-law plus a Gaussian peak in primary mass, wherein the spin distributions of the two components in mass are allowed to differ from each other. In our case, we replace the power law mass components and parametric spin descriptions with fully data-driven (often called "non-parametric") Basis-Spline functions. We choose to describe peaks as Gaussian in log-mass, i.e., as log-normal components in the mass distribution.

We make use of the mass and spin Basis-Spline (B-Spline) models from \cite{2022arXiv221012834E}, with a few modifications. We fix the number of knots $n$ in all B-Spline models used, choosing $n_{m_1}=48$, $n_{q}=30$, $n_a=16$, and $n_{cos(\theta)}=16$. When used in a bayesian inference setting, a prior can be used to place a smoothing requirement on the B-Spline function by penalizing large differences between neighboring coefficients. This penalization allows the user to choose a large number of basis functions (i.e. coefficients) to accurrately fit the data without concern for overfitting. Our coefficient and smoothing priors have the same form as in equations B4 and B5 in \brucepaper{}, though we fix the value of the smoothing prior scale $\tau_\lambda$ for each B-Spline model. 

\remove{We found that fitting $\tau_\lambda$ with the prior specified in Table 3 of \brucepaper{}, namely a Uniform distribution, exhibited railing on the lower bound of the distribution. This indicated a potential issue with how the smoothing penalization should be specified for CBC population inference. Our choice in this work was to fix $\tau_\lambda$ to a value that reasonably produces distributions similar to other works like \brucepaper and \citet{10.48550/arXiv.2302.07289}.} \jaxen{REPLACE WITH: We encountered low effective sample sizes when marginalizing over the smoothing prior scales, $\tau_\lambda$, so we opted to fix $\tau_\lambda$ for each population distribution to values high enough to avoid this issue while still producing results consistent with other non-parametric results like \citet{2022ApJ...924..101E}, \brucepaper{}, and \citet{10.48550/arXiv.2302.07289}.}

We plan to address this issue in a future work by exploring different coefficient and smoothing prior specifications, such as locally adaptive smoothing prio. For a full description of the priors used in each model, see Appendix \ref{sec:priors}.

In both of our model prescriptions, detailed below, the spin magnitude and tilt distributions of each binary component are assumed to be independently and identically distributed (IID); i.e. both binary components have spins drawn from the same (inferred) distribution. We do not fit the mininum and maximum BBH primary mass, instead truncating all the mass distributions below $m_\text{min} = 5.0\msun$ and above $m_\text{max} = 100\msun$; however, estimates of minimum and maximum BBH mass are still possible as B-Splines are able to become arbitarily small in locations where the data requires it, effectively placing upper bounds on merger rate densities in regions without detections, consistent with the observed $VT$. Finally, we estimate the redshift distribution as in \brucepaper, with a modulated powerlaw distribution, and assume it is the same for each subpopulation.

To construct our probabilistic mixture model, we sample $p_{\lambda} \sim \mathcal{D}(M)$, from an M-dimensional Dirichlet distribution of equal weights, representing the astrophysical branching ratios $\lambda_{i}$ of each subpopulation. The total population model $p(\theta | \{\Lambda\})$ is then a weighted sum of each subpopulation model $p_i(\theta|\Lambda_i)$ with $\lambda_i$ as the weights:

\begin{equation} \label{totmixmod}
p_\text{tot}(\theta|\{\Lambda\}) = \sum_{i=0}^{M} \lambda_i p_i(\theta | \Lambda_i)
\end{equation}

Within the \textsc{NumPyro} \citep{1810.09538,1912.11554} probabilistic programming language, we use the \texttt{NUTS} \citep{1111.4246} sampler to perform our parameter estimation. All the models and formalism used in our analysis are available in the \href{https://github.com/FarrOutLab/GWInferno}{GWInferno} python library, along with the code and data to reproduce this study in this GitHub \href{https://github.com/jaxeng/CosmicCousins}{repository}. 

\subsubsection{\base{}}
In this model prescription, we categorize the BBH population into $M=2$ subpopulations, which we will refer to as \first{} and \contB{} based on their primary mass distributions (continuum here referring to the data-driven nature of the B-Spline distributions). The \first{} subpopulation is characterised by a truncated Log-Normal peak in primary mass and B-Spline functions in mass ratio, spin magnitude, and tilt angle. The \contB{} subpopulation is characterized by B-Spline functions in all mass and spin parameters. We infer the mean $\mu_m$ and standard deviation $\sigma_m$ of the LogNormal peak, along with the B-Spline coefficients $\mathbf{c}$ for all other parameters.

\begin{itemize}
    \item \first{} (64 parameters), i = 0. This category assumes a truncated Log-Normal model in primary mass and B-spline ($B_k$) models in mass ratio $q$, spin magnitude $a_j$, and $cos(\theta_j)$. Note that since we assume the spins to be IID, the B-Spline spin parameters, such as $\mathbf{c}_{a,0}$, are the same for each binary component $j=1$ and $j=2$
    \begin{eqnarray} \label{eq:peakAbase}
        p_{m,0}(m_1| \Lambda_{m,0}) = \text{Lognormal}_\text{T}(m_1 | \mu_{m}, \sigma_{m}) \\
        \text{log} p_{q,0}(q| \Lambda_{q,0}) = B_k(q | \mathbf{c}_{q,0}) \\
        \text{log} p_{a,0}(a_j| \Lambda_{a,0}) = B_k(a_j | \mathbf{c}_{a,0}) \\
        \text{log} p_{\theta,0}(cos(\theta_j)| \Lambda_{\theta,0}) = B_k( cos(\theta_j) | \mathbf{c}_{\theta,0})
    \end{eqnarray}

    \item \contB{} (110 parameters), $i=1$. The mass ratio and spin models have the same form as \first{}, but here primary mass is fit to a B-spline function. 
    \begin{eqnarray} \label{eq:contBbase}
        \text{log} p_{m,1}(m_1| \Lambda_{m,1}) = B_k(m_1 | \mathbf{c}_{m}) \\
        \text{log} p_{q,1}(q| \Lambda_{q,0}) = B_k(q | \mathbf{c}_{q,1}) \\
        \text{log} p_{a,1}(a_j| \Lambda_{a,1}) = B_k(a_j | \mathbf{c}_{a,1}) \\
        \text{log} p_{\theta,1}(cos(\theta_j)| \Lambda_{\theta,1}) = B_k( cos(\theta_j) | \mathbf{c}_{\theta,1})
    \end{eqnarray}

\end{itemize}

\subsubsection{\comp{}}

To investigate whether other parts of the primary mass spectrum may have similar spin characteristics to \first{}, we construct a composite subpopulation model, wherein we include an additional mass component \contA{}, described by a B-Spline in primary mass but force it to have the same spin distributions as \first{}. \contB{} is still included and is described by it's own spin distributions.

\begin{itemize}
    \item \first{} (64 parameters), $i=0$. This category assumes a truncated Log-Normal model in primary mass, and B-spline models in mass ratio $q$, spin magnitude $a_j$, and $cos(\theta_{j})$. 
    \begin{eqnarray} \label{eq:peakAcomp}
        p_{m,0}(m_1| \Lambda_{m,0}) = \text{Lognormal}_\text{T}(m_1 | \mu_{m}, \sigma_{m}) \\
        \text{log} p_{q,0}(q| \Lambda_{q,0}) = B_k(q | \mathbf{c}_{q,0}) \\
        \text{log} p_{a,0}(a_j| \Lambda_{a,0}) = B_k(a_j | \mathbf{c}_{a,0}) \\
        \text{log} p_{\theta,0}(cos(\theta_j)| \Lambda_{\theta,0}) = B_k( cos(\theta_j) | \mathbf{c}_{\theta,0})
    \end{eqnarray}

    \item \contA{} (78 parameters), $i=1$ for mass, $i = 0$ for spin. The mass ratio and spin models are the same as the previous category, but here primary mass is fit to a B-spline function. \textit{Note: the branching fraction for this model is $\lambda_1$, which is allowed to differ from $\lambda_0$.}
    \begin{eqnarray} \label{eq:contAcomp}
        \text{log} p_{m,1}(m_1| \Lambda_{m,1}) = B_k(m_1 | \mathbf{c}_{m, 1}) \\
        \text{log} p_{q,1}(q| \Lambda_{q,1}) = B_k(q | \mathbf{c}_{q,1}) \\
        \text{log} p_{a,1}(a_j| \Lambda_{a,1}) = B_k(a_j | \mathbf{c}_{a,0}) \\
        \text{log} p_{\theta,1}(cos(\theta_j)| \Lambda_{\theta,1}) = B_k( cos(\theta_j) | \mathbf{c}_{\theta,0})
    \end{eqnarray}

    \item \contB{} (110 parameters), $i=2$. Here, all mass and spin properties are fit to B-Splines.
    \begin{eqnarray} \label{eq:contBcomp}
        \text{log} p_{m,2}(m_1| \Lambda_{m,2}) = B_k(m_1 | \mathbf{c}_{m, 2}) \\
        \text{log} p_{q,2}(q| \Lambda_{q,2}) = B_k(q | \mathbf{c}_{q,2}) \\
        \text{log} p_{a,2}(a_j| \Lambda_{a,2}) = B_k(a_j | \mathbf{c}_{a,2}) \\
        \text{log} p_{\theta,2}(cos(\theta_j)| \Lambda_{\theta,2}) = B_k( cos(\theta_j) | \mathbf{c}_{\theta,2})
    \end{eqnarray}
\end{itemize}