\documentclass[twocolumn, linenumbers]{aastex631}

\newcommand{\msun}{\ensuremath{{\rm M}_\odot}}
\newcommand{\result}[1]{\textcolor{red}{#1}}
\newcommand{\bruce}[1]{\textcolor{blue}{BE: #1}}
\newcommand{\NewChange}[1]{\textcolor{green}{#1}}

\newcommand{\CIPlusMinus}[1]{{#1[median]^{+#1[error plus]}_{-#1[error minus]}}}
\newcommand{\CIBoundsBracket}[1]{{[#1[5th percentile], #1[95th percentile]]}}
\newcommand{\CIBoundsDash}[1]{{#1[5th percentile]\textendash#1[95th percentile]}}

\graphicspath{{./}{figures/}}

\usepackage{showyourwork}
\usepackage{bm}
\usepackage{amsmath}
\usepackage{appendix}


\begin{document}

\title{Paper title}

\author{Jaxen Godfrey}
\email{jaxeng@uoregon.edu}
\affiliation{Institute  for  Fundamental  Science, Department of Physics, University of Oregon, Eugene, OR 97403, USA}

\author{Bruce Edelman}
\affiliation{Institute  for  Fundamental  Science, Department of Physics, University of Oregon, Eugene, OR 97403, USA}

\author{Ben Farr}
\affiliation{Institute  for  Fundamental  Science, Department of Physics, University of Oregon, Eugene, OR 97403, USA}

% Abstract with filler text
\begin{abstract}
    Lorem ipsum dolor sit amet, consectetuer adipiscing elit.
    Ut purus elit, vestibulum ut, placerat ac, adipiscing vitae, felis.
    Curabitur dictum gravida mauris, consectetuer id, vulputate a, magna.
    Donec vehicula augue eu neque, morbi tristique senectus et netus et.
    Mauris ut leo, cras viverra metus rhoncus sem, nulla et lectus vestibulum.
    Phasellus eu tellus sit amet tortor gravida placerat.
    Integer sapien est, iaculis in, pretium quis, viverra ac, nunc.
    Praesent eget sem vel leo ultrices bibendum.
    Aenean faucibus, morbi dolor nulla, malesuada eu, pulvinar at, mollis ac.
    Curabitur auctor semper nulla donec varius orci eget risus.
    Duis nibh mi, congue eu, accumsan eleifend, sagittis quis, diam.
    Duis eget orci sit amet orci dignissim rutrum.
\end{abstract}

% Main body with filler text
\section{Introduction}
\label{sec:intro}
test
Lorem ipsum dolor sit amet, consectetuer adipiscing elit.
Ut purus elit, vestibulum ut, placerat ac, adipiscing vitae, felis.
Curabitur dictum gravida mauris, consectetuer id, vulputate a, magna.
Donec vehicula augue eu neque, morbi tristique senectus et netus et.
Mauris ut leo, cras viverra metus rhoncus sem, nulla et lectus vestibulum.
Phasellus eu tellus sit amet tortor gravida placerat.
Integer sapien est, iaculis in, pretium quis, viverra ac, nunc.
Praesent eget sem vel leo ultrices bibendum.
Aenean faucibus, morbi dolor nulla, malesuada eu, pulvinar at, mollis ac.
Curabitur auctor semper nulla donec varius orci eget risus.
Duis nibh mi, congue eu, accumsan eleifend, sagittis quis, diam.
Duis eget orci sit amet orci dignissim rutrum.

\begin{figure}[ht!]
    \script{random_numbers.py}
    \begin{centering}
        \includegraphics[width=\linewidth]{figures/random_numbers.pdf}
        \caption{
            Plot showing a bunch of random numbers.
        }
        \label{fig:random_numbers}
    \end{centering}
\end{figure}

Nam dui ligula, fringilla a, euismod sodales, sollici- tudin vel, wisi.
Morbi auctor lorem non justo, nam lacus libero, pretium at, lobortis vitae.
Donec aliquet, tortor sed accumsan bibendum, erat ligula aliquet magna.
Morbi ac orci et nisl hendrerit mollis, suspendisse ut massa, cras nec ante.
Pellentesque a nulla cum sociis natoque penatibus et magnis dis parturient.
Aliquam tincidunt urna, nulla ullamcorper vestibulum turpis.
Pellentesque cursus luctus mauris \citep{Luger2021}.


\section{Methods} \label{sec:methods}

\begin{itemize}
    \item VERY quick basic point to hierarchical Bayesian inference details
    \item Introduce discrete latent variables / marginalized mixture models
    \item describe equivalence of fitting discrete latent variables to marginalized mixture models
    \item point out GWInferno and code etc
    \item describe the specific models we use in the paper
\end{itemize}
\section{Results} \label{sec:results}

\begin{figure*}[ht!]
    \begin{centering}
        \includegraphics[width=\linewidth]{figures/mass_distribution_plot.pdf}
        \caption{The marginal primary mass distribution}
        \label{fig:mass_distribution}
    \end{centering}
    \script{mass_distribution_plot.py}
\end{figure*}

\begin{figure*}[ht!]
    \begin{centering}
        \includegraphics[width=\linewidth]{figures/spin_mag_distribution_plot.pdf}
        \caption{The marginal primary spin magnitude distribution}
        \label{fig:spin_mag_distribution}
    \end{centering}
    \script{spin_distributions_plot.py}
\end{figure*}

\begin{itemize}
    \item Start by introducing the dataset (GWTC-3) and threshold/cuts on catalog for our dataset
    \item Show results of main run model -- mass dist -- spin dists etc
    \item Discuss more specific details on different subpopulation mass/spin dists
    \item Talk about astrophysical branching ratios of subpopulations and which observations were "put" within each of the subpops
    \item Quantitative statements on spin mag dist of our isolated subpopulation
    \item Quantitative statements on spin orientation dist of our isolated subpop. How much does it prefer aligned spins over the other subpops?
\end{itemize}

With these models and framework in hand, we infer the mass and spin distributions with the recently released LVK catalog of gravitational wave observations, GWTC-3 \jaxen{CITE THIS}. We perform the same BBH threshold cuts on the catalog done by the LVK's accompnaying population analysis \jaxen{CITE THIS}, which leaves us with 70 BBH mergers. Additionally, we choose to remove GW190814 from our anlaysis, as it is likely to be an outlier of the total BBH population and is not very well understood \jaxen{CITE THIS}. With the 69 remaining events, we are able to infer the mass and spin distributions of three potential BBH subpopulations, detailed below. 
$\CIPlusMinus{\macros[Mass][LowMassPeak]}\,\msun$

\subsection{BBH Mass and Spin Distributions}

Figure \ref{fig:mass_distribution} shows the inferred total primary mass distribution, as well as the distributions of the three subpopulations, \textsc{Low-Mass Peak}, \textsc{Mid-Mass Peak}, and \textsc{Continuum}. For comparison, it also includes the primary mass distributions inferred by \jaxen{CITE LVK GWTC-3 POP} and \jaxen{CITE BRUCE}. Figure 

\begin{figure}[ht!]
    \begin{centering}
        \includegraphics[width=\linewidth]{figures/radar_plot.pdf}
        \caption{}
        \label{fig:radar_plot}
    \end{centering}
    \script{radar_plot.py}
\end{figure}
\section{Astrophysical Interpretation} \label{sec:astro}

% \begin{itemize}
%     \item What can this new identified subpop help to enlighten in stellar pop synth community?
%     \item can we use spin tilt dist to make statements on supernovae kicks in isolated formation?
%     \item How does this compare to LVK work and other recent work? Are our results consistent or in conflict with dyn/iso fractions?
%     \item report fdyn / fhm and etc
% \end{itemize}

Spin predictions of binaries formed in the stellar field are dependent on the many physical proccesses that may occur prior to the stellar binary becoming a BBH.  Spin magnitude of a BH can depend largely on the efficiency of angular monentum (AM) transport between it's progenitor stellar core and envelope \citep{2203.02515}. Efficient AM transport, such as through the Taylor-Spruit magnetic dynamo \citep{10.1051/0004-6361:20011465}, leads to low spinning BHs \citep{10.3847/2041-8213/ab339b} while less efficient AM transport, such as that predicted by the shellular model \citep{1992A&A...265..115Z,2012A&A...537A.146E,10.3847/1538-4365/aacb24,2019MNRAS.485.4641C}, can preserve the spin of the pregenitor star. Howver, the effects of AM transport by these mechanisms can be obfuscated by tidal interactions between the binary components, accretion, and mass tranfser, which can spin up the binary or increase AM transport efficiency, thus shedding spin. Natal supernova kicks are thought to be the leading cause of spin orbit misalignments in field binaries, while tidal forces and mass transfer tend to align BH spins with the orbital AM. While we believe the $10\msun$ subpopulation is consistent with the general characteristics associated with field formation, large uncertainites in both spin measurements and predictions from population synthesis models prevent us from placing informative constraints on the formation physics. If the $10\msun$ subpopulation is indeed a product of isolated binary evolution, then our inferred spin distributions hint at this channel producing binaries with low, modestly misaligned spins. This could indicate that AM transport is efficient in massive stars, as modeled by variations of the Taylor-Spruit magnetic dynamo \citep{1706.07053}, and that natal kicks are a common occurence during field BBH formation. Our results are consistent with other analyses that find the data does not require a zero-spin subpopulation \citep{arXiv2205.08574,2301.01312}, though is in tension with other analyses that have claimed its existence \citep{doi.org/10.3847/2041-8213/ac2f3c,2105.10580}; however, we cannot completely rule out zero-spin.

The $35\msun$ peak may also be consistent with field formation, as we infer spin properties consistent to those of the $10\msun$ peak. The sharp fall-off in primary mass of \contA{} in the \comp{} model could give an estimate of the lower edge of the PISN mass gap. The 99th percentile primary mass of \contA{} is $m_{1,99\%} = $ \result{$\CIPlusMinus{\macros[Mass][Composite][ContinuumA][99percentile]}$ \msun}, which is consistent with predictions that place the lower edge of the gap between $40-70\msun$ \citep{1901.00215,1910.12874v1,2103.07933v1,2104.07783v2}. 

\section{Conclusion} \label{sec:conclusion}

% \begin{itemize}
%     \item Discuss further work on other ways we can use this method to probe formation channels even deeper. (use spin vs mass dist to disentangle the sub pops. i.e. isotropic tilt for dynamical -- aligned tilt for isolated)
% \end{itemize}

As the catalog of compact object mergers continues to grow, we are able to probe the physical properties of these systems with higher fidelity and uncover details in their distributions previously obscured by our lack of data. With these advancements comes the ability to piece together formation histories imprinted in the details of these distributions. Understanding the physical properties of CBC's and their formation has implications for the broader astrophysics community such as providing constraints on stellar evolution theories and population synthesis simulations, the physics of globular clusters, stellar metalicity, neutron star equations of state, and much more. 

By leveraging the hierarchical Bayesian inference toolkit, a mixture of parametric and non-parametric models, and combining information across mass and spin, we were able to identify a peak in the BBH primary mass spectrum at $m_\text{1,peak} = $ \result{$\CIPlusMinus{\macros[Mass][Base][PeakA][max]}$ \msun} that corresponds to a subpopulation of BBH's with low spins and a moderate preference for alignment, consistent with isolated binary formation. We then extended our \base{} to search the rest of the mass spectrum for events with similar spin characteristics to the $10\msun$ subpopulation. We found that the peak in the mass spectrum near $35\msun$ was consistent with the $10\msun$ events. The categorization of the $20\msun$ events remains unclear, since $\sim20\%$ of posterior samples categorized the $20\msun$ peak with the highest mass events. It is also unclear whether the spin tilt distribution of the high mass events is an informed isotropic distribution or just uninformed due to large measurement uncertainty. 

Due to the large uncertainities that are currently present in measurements of BBH properties and population synthesis models, we are unable to place strong constraints on the physics behind isolated binary evolution or (P)PISN. However, if our $10\msun$ and $35\msun$ subpopulations are truly products of these channels, they likely produce binaries with low spins, though the subpopulations being dominated by zeros spin is not ruled out. Aligned spin systems are also not completely ruled out but the subpopulations appear to possess modest misalignments and the apparent fall-off of the $35\msun$ peak may indicate a lower bound on the PISN mass gap of \result{$\CIPlusMinus{\macros[Mass][Composite][ContinuumA][99percentile]}$ \msun}.

The discrete latent variable framework laid out in this analysis and developed in the python library \textsc{GWInferno} can be used to understand the full CBC catalog beyond identifying BBH subpopulations. Currently the LVK categorizes mergers as binary black holes, binary neutron stars, or neutron star binary black holes a-priori based on mass thresholds and tidal deformation values. Instead of categorizing merger components a-priori and then fitting the mass and spin distributions of each category individually, discrete latent variables could be used to simultaneously classify merger components and infer their mass and spin distributions.

% In future work, we would like to incorporate a method for classifying outliers to subpopulations as well as testing this framework on a simulated catalog in order to better understand how uninformed data affects our results. Repeating this analysis on the data from the next observing run, which is expected to increase the catalog size tenfold \jaxen{CITE}, should provide more insightful results with stronger constraints on subpopulation properties. 

In future work, we would like to incorporate a method for classifying outliers to subpopulations as well as testing this framework on a simulated catalog in order to better understand how uninformed data affects our results. Repeating this analysis on the data from the next observing run, which is expected to increase the catalog size tenfold \jaxen{CITE}, should provide more insightful results with stronger constraints on subpopulation properties. 

\section{Acknowledgements}\label{sec:acknowledments}
This research has made use of data, software and/or web tools obtained from the Gravitational Wave Open Science Center 
(\url{https://www.gw-openscience.org/}), a service of LIGO Laboratory, the LIGO Scientific Collaboration and the Virgo Collaboration. 
The authors are grateful for computational resources provided by the LIGO Laboratory and supported by National Science Foundation Grants PHY-0757058 and PHY-0823459.  
This work benefited from access to the University of Oregon high performance computer, Talapas. This material is based upon work supported 
in part by the National Science Foundation under Grant PHY-1807046 and work supported by NSF’s LIGO Laboratory which is a major facility 
fully funded by the National Science Foundation.
\software{
\textsc{Astropy}~\citep{2018AJ....156..123A},
\textsc{NumPy}~\citep{harris2020array},
\textsc{SciPy}~\citep{2020SciPy-NMeth},
\textsc{Matplotlib}~\citep{Hunter:2007},
\textsc{Jax}~\citep{jax},
\textsc{NumPyro}~\citep{pyro,numpyro},
}
\bibliography{bib}{}
\bibliographystyle{aasjournal}

\end{document}