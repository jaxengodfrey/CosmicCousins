\PassOptionsToPackage{prologue,dvipsnames}{xcolor}
\documentclass[twocolumn]{aastex631}

\usepackage{showyourwork}
\usepackage{bm}
\usepackage{amsmath}
\usepackage{booktabs}
\usepackage{appendix}
\usepackage{ifthen}
\usepackage{array}
\usepackage{soul}
\usepackage{float}
\makeatletter\newcommand\macros[1][all]{\ifnum\pdfstrcmp{#1}{all}=0\def\macros@out{\{"Mass": \{"LowMassPeak": \{"1percentile": \{"median": 9.5, "error plus": 1.1, "error minus": 1.8, "5th percentile": 7.7, "95th percentile": 11, "10th percentile": 8.2, "90th percentile": 10\}, "99percentile": \{"median": 12, "error plus": 1.3, "error minus": 0.95, "5th percentile": 11, "95th percentile": 13, "10th percentile": 11, "90th percentile": 13\}, "max": \{"median": 11, "error plus": 0.83, "error minus": 0.95, "5th percentile": 9.8, "95th percentile": 12, "10th percentile": 10.0, "90th percentile": 11\}\}, "HighMassPeak": \{"1percentile": \{"median": 6.0, "error plus": 5.6, "error minus": 0.71, "5th percentile": 5.2, "95th percentile": 12, "10th percentile": 5.4, "90th percentile": 9.6\}, "99percentile": \{"median": 50, "error plus": 7.1, "error minus": 6.7, "5th percentile": 44, "95th percentile": 57, "10th percentile": 45, "90th percentile": 56\}, "max": \{"median": 24, "error plus": 5.6, "error minus": 9.2, "5th percentile": 15, "95th percentile": 29, "10th percentile": 16, "90th percentile": 29\}\}, "Continuum": \{"1percentile": \{"median": 11, "error plus": 16, "error minus": 4.8, "5th percentile": 6.3, "95th percentile": 27, "10th percentile": 6.8, "90th percentile": 22\}, "99percentile": \{"median": 90, "error plus": 9.5, "error minus": 26, "5th percentile": 64, "95th percentile": 100, "10th percentile": 73, "90th percentile": 99\}, "max": \{"median": 27, "error plus": 53, "error minus": 19, "5th percentile": 8.4, "95th percentile": 80, "10th percentile": 9.5, "90th percentile": 69\}\}\}, "SpinMag": \{"LowMassPeak": \{"1percentile": \{"median": 0.0067, "error plus": 0.01, "error minus": 0.0033, "5th percentile": 0.0033, "95th percentile": 0.017, "10th percentile": 0.0033, "90th percentile": 0.013\}, "99percentile": \{"median": 0.98, "error plus": 0.011, "error minus": 0.031, "5th percentile": 0.95, "95th percentile": 0.99, "10th percentile": 0.96, "90th percentile": 0.99\}, "max": \{"median": 0.16, "error plus": 0.19, "error minus": 0.16, "5th percentile": 0.0, "95th percentile": 0.35, "10th percentile": 0.023, "90th percentile": 0.3\}\}, "HighMassPeak": \{"1percentile": \{"median": 0.0078, "error plus": 0.013, "error minus": 0.0044, "5th percentile": 0.0033, "95th percentile": 0.021, "10th percentile": 0.0044, "90th percentile": 0.017\}, "99percentile": \{"median": 0.98, "error plus": 0.01, "error minus": 0.024, "5th percentile": 0.96, "95th percentile": 0.99, "10th percentile": 0.96, "90th percentile": 0.99\}, "max": \{"median": 0.23, "error plus": 0.28, "error minus": 0.23, "5th percentile": 0.0, "95th percentile": 0.51, "10th percentile": 0.0033, "90th percentile": 0.45\}\}, "Continuum": \{"1percentile": \{"median": 0.011, "error plus": 0.018, "error minus": 0.0067, "5th percentile": 0.0044, "95th percentile": 0.029, "10th percentile": 0.0044, "90th percentile": 0.023\}, "99percentile": \{"median": 0.99, "error plus": 0.0067, "error minus": 0.017, "5th percentile": 0.97, "95th percentile": 0.99, "10th percentile": 0.98, "90th percentile": 0.99\}, "max": \{"median": 0.49, "error plus": 0.51, "error minus": 0.49, "5th percentile": 0.0, "95th percentile": 1.0, "10th percentile": 0.011, "90th percentile": 0.94\}\}\}, "CosTilt": \{"LowMassPeak": \{"1percentile": \{"median": {-}0.97, "error plus": 0.047, "error minus": 0.02, "5th percentile": {-}0.99, "95th percentile": {-}0.92, "10th percentile": {-}0.98, "90th percentile": {-}0.93\}, "99percentile": \{"median": 0.98, "error plus": 0.013, "error minus": 0.031, "5th percentile": 0.95, "95th percentile": 0.99, "10th percentile": 0.96, "90th percentile": 0.99\}, "max": \{"median": 0.49, "error plus": 0.51, "error minus": 0.82, "5th percentile": {-}0.33, "95th percentile": 1.0, "10th percentile": {-}0.0056, "90th percentile": 0.9\}, "log10gammafrac": \{"median": 0.24, "error plus": 0.5, "error minus": 0.5, "5th percentile": {-}0.26, "95th percentile": 0.74, "10th percentile": {-}0.15, "90th percentile": 0.62\}, "negfrac": \{"median": 0.35, "error plus": 0.14, "error minus": 0.12, "5th percentile": 0.22, "95th percentile": 0.49, "10th percentile": 0.25, "90th percentile": 0.46\}\}, "HighMassPeak": \{"1percentile": \{"median": {-}0.97, "error plus": 0.044, "error minus": 0.016, "5th percentile": {-}0.99, "95th percentile": {-}0.93, "10th percentile": {-}0.98, "90th percentile": {-}0.94\}, "99percentile": \{"median": 0.98, "error plus": 0.011, "error minus": 0.027, "5th percentile": 0.95, "95th percentile": 0.99, "10th percentile": 0.96, "90th percentile": 0.99\}, "max": \{"median": 0.2, "error plus": 0.8, "error minus": 0.77, "5th percentile": {-}0.58, "95th percentile": 1.0, "10th percentile": {-}0.34, "90th percentile": 0.84\}, "log10gammafrac": \{"median": 0.15, "error plus": 0.46, "error minus": 0.45, "5th percentile": {-}0.3, "95th percentile": 0.61, "10th percentile": {-}0.2, "90th percentile": 0.51\}, "negfrac": \{"median": 0.44, "error plus": 0.12, "error minus": 0.13, "5th percentile": 0.31, "95th percentile": 0.57, "10th percentile": 0.34, "90th percentile": 0.54\}\}, "Continuum": \{"1percentile": \{"median": {-}0.98, "error plus": 0.036, "error minus": 0.013, "5th percentile": {-}0.99, "95th percentile": {-}0.94, "10th percentile": {-}0.99, "90th percentile": {-}0.95\}, "99percentile": \{"median": 0.98, "error plus": 0.013, "error minus": 0.031, "5th percentile": 0.94, "95th percentile": 0.99, "10th percentile": 0.95, "90th percentile": 0.99\}, "max": \{"median": {-}0.081, "error plus": 1.1, "error minus": 0.92, "5th percentile": {-}1.0, "95th percentile": 1.0, "10th percentile": {-}0.94, "90th percentile": 0.83\}, "log10gammafrac": \{"median": {-}0.025, "error plus": 0.54, "error minus": 0.52, "5th percentile": {-}0.54, "95th percentile": 0.52, "10th percentile": {-}0.43, "90th percentile": 0.4\}, "negfrac": \{"median": 0.5, "error plus": 0.19, "error minus": 0.19, "5th percentile": 0.31, "95th percentile": 0.69, "10th percentile": 0.35, "90th percentile": 0.65\}\}\}, "BranchingRatios": \{"LowMassPeak": \{"Frac": \{"median": 0.77, "error plus": 0.11, "error minus": 0.17\}, "Percent": \{"median": 77, "error plus": 11, "error minus": 17\}\}, "HighMassPeak": \{"Frac": \{"median": 0.21, "error plus": 0.16, "error minus": 0.1\}, "Percent": \{"median": 21, "error plus": 16, "error minus": 10\}\}, "Continuum": \{"Frac": \{"median": 0.011, "error plus": 0.058, "error minus": 0.0096\}, "Percent": \{"median": 1.1, "error plus": 5.8, "error minus": 0.96\}\}\}, "NumEvents": \{"LowMassPeak": \{"median": 15, "error plus": 0.0, "error minus": 2.0\}, "HighMassPeak": \{"median": 50, "error plus": 4.0, "error minus": 13\}, "Continuum": \{"median": 5.0, "error plus": 13, "error minus": 4.0\}\}\}}\else\ifnum\pdfstrcmp{#1}{Mass}=0\let\macros@out\macros@I\else\ifnum\pdfstrcmp{#1}{SpinMag}=0\let\macros@out\macros@II\else\ifnum\pdfstrcmp{#1}{CosTilt}=0\let\macros@out\macros@III\else\ifnum\pdfstrcmp{#1}{BranchingRatios}=0\let\macros@out\macros@IV\else\ifnum\pdfstrcmp{#1}{NumEvents}=0\let\macros@out\macros@V\else\def\macros@out{??}\fi\fi\fi\fi\fi\fi\macros@out}\newcommand\macros@I[1][all]{\ifnum\pdfstrcmp{#1}{all}=0\def\macros@I@out{\{"LowMassPeak": \{"1percentile": \{"median": 9.5, "error plus": 1.1, "error minus": 1.8, "5th percentile": 7.7, "95th percentile": 11, "10th percentile": 8.2, "90th percentile": 10\}, "99percentile": \{"median": 12, "error plus": 1.3, "error minus": 0.95, "5th percentile": 11, "95th percentile": 13, "10th percentile": 11, "90th percentile": 13\}, "max": \{"median": 11, "error plus": 0.83, "error minus": 0.95, "5th percentile": 9.8, "95th percentile": 12, "10th percentile": 10.0, "90th percentile": 11\}\}, "HighMassPeak": \{"1percentile": \{"median": 6.0, "error plus": 5.6, "error minus": 0.71, "5th percentile": 5.2, "95th percentile": 12, "10th percentile": 5.4, "90th percentile": 9.6\}, "99percentile": \{"median": 50, "error plus": 7.1, "error minus": 6.7, "5th percentile": 44, "95th percentile": 57, "10th percentile": 45, "90th percentile": 56\}, "max": \{"median": 24, "error plus": 5.6, "error minus": 9.2, "5th percentile": 15, "95th percentile": 29, "10th percentile": 16, "90th percentile": 29\}\}, "Continuum": \{"1percentile": \{"median": 11, "error plus": 16, "error minus": 4.8, "5th percentile": 6.3, "95th percentile": 27, "10th percentile": 6.8, "90th percentile": 22\}, "99percentile": \{"median": 90, "error plus": 9.5, "error minus": 26, "5th percentile": 64, "95th percentile": 100, "10th percentile": 73, "90th percentile": 99\}, "max": \{"median": 27, "error plus": 53, "error minus": 19, "5th percentile": 8.4, "95th percentile": 80, "10th percentile": 9.5, "90th percentile": 69\}\}\}}\else\ifnum\pdfstrcmp{#1}{LowMassPeak}=0\let\macros@I@out\macros@VI\else\ifnum\pdfstrcmp{#1}{HighMassPeak}=0\let\macros@I@out\macros@VII\else\ifnum\pdfstrcmp{#1}{Continuum}=0\let\macros@I@out\macros@VIII\else\def\macros@I@out{??}\fi\fi\fi\fi\macros@I@out}\newcommand\macros@II[1][all]{\ifnum\pdfstrcmp{#1}{all}=0\def\macros@II@out{\{"LowMassPeak": \{"1percentile": \{"median": 0.0067, "error plus": 0.01, "error minus": 0.0033, "5th percentile": 0.0033, "95th percentile": 0.017, "10th percentile": 0.0033, "90th percentile": 0.013\}, "99percentile": \{"median": 0.98, "error plus": 0.011, "error minus": 0.031, "5th percentile": 0.95, "95th percentile": 0.99, "10th percentile": 0.96, "90th percentile": 0.99\}, "max": \{"median": 0.16, "error plus": 0.19, "error minus": 0.16, "5th percentile": 0.0, "95th percentile": 0.35, "10th percentile": 0.023, "90th percentile": 0.3\}\}, "HighMassPeak": \{"1percentile": \{"median": 0.0078, "error plus": 0.013, "error minus": 0.0044, "5th percentile": 0.0033, "95th percentile": 0.021, "10th percentile": 0.0044, "90th percentile": 0.017\}, "99percentile": \{"median": 0.98, "error plus": 0.01, "error minus": 0.024, "5th percentile": 0.96, "95th percentile": 0.99, "10th percentile": 0.96, "90th percentile": 0.99\}, "max": \{"median": 0.23, "error plus": 0.28, "error minus": 0.23, "5th percentile": 0.0, "95th percentile": 0.51, "10th percentile": 0.0033, "90th percentile": 0.45\}\}, "Continuum": \{"1percentile": \{"median": 0.011, "error plus": 0.018, "error minus": 0.0067, "5th percentile": 0.0044, "95th percentile": 0.029, "10th percentile": 0.0044, "90th percentile": 0.023\}, "99percentile": \{"median": 0.99, "error plus": 0.0067, "error minus": 0.017, "5th percentile": 0.97, "95th percentile": 0.99, "10th percentile": 0.98, "90th percentile": 0.99\}, "max": \{"median": 0.49, "error plus": 0.51, "error minus": 0.49, "5th percentile": 0.0, "95th percentile": 1.0, "10th percentile": 0.011, "90th percentile": 0.94\}\}\}}\else\ifnum\pdfstrcmp{#1}{LowMassPeak}=0\let\macros@II@out\macros@IX\else\ifnum\pdfstrcmp{#1}{HighMassPeak}=0\let\macros@II@out\macros@X\else\ifnum\pdfstrcmp{#1}{Continuum}=0\let\macros@II@out\macros@XI\else\def\macros@II@out{??}\fi\fi\fi\fi\macros@II@out}\newcommand\macros@III[1][all]{\ifnum\pdfstrcmp{#1}{all}=0\def\macros@III@out{\{"LowMassPeak": \{"1percentile": \{"median": {-}0.97, "error plus": 0.047, "error minus": 0.02, "5th percentile": {-}0.99, "95th percentile": {-}0.92, "10th percentile": {-}0.98, "90th percentile": {-}0.93\}, "99percentile": \{"median": 0.98, "error plus": 0.013, "error minus": 0.031, "5th percentile": 0.95, "95th percentile": 0.99, "10th percentile": 0.96, "90th percentile": 0.99\}, "max": \{"median": 0.49, "error plus": 0.51, "error minus": 0.82, "5th percentile": {-}0.33, "95th percentile": 1.0, "10th percentile": {-}0.0056, "90th percentile": 0.9\}, "log10gammafrac": \{"median": 0.24, "error plus": 0.5, "error minus": 0.5, "5th percentile": {-}0.26, "95th percentile": 0.74, "10th percentile": {-}0.15, "90th percentile": 0.62\}, "negfrac": \{"median": 0.35, "error plus": 0.14, "error minus": 0.12, "5th percentile": 0.22, "95th percentile": 0.49, "10th percentile": 0.25, "90th percentile": 0.46\}\}, "HighMassPeak": \{"1percentile": \{"median": {-}0.97, "error plus": 0.044, "error minus": 0.016, "5th percentile": {-}0.99, "95th percentile": {-}0.93, "10th percentile": {-}0.98, "90th percentile": {-}0.94\}, "99percentile": \{"median": 0.98, "error plus": 0.011, "error minus": 0.027, "5th percentile": 0.95, "95th percentile": 0.99, "10th percentile": 0.96, "90th percentile": 0.99\}, "max": \{"median": 0.2, "error plus": 0.8, "error minus": 0.77, "5th percentile": {-}0.58, "95th percentile": 1.0, "10th percentile": {-}0.34, "90th percentile": 0.84\}, "log10gammafrac": \{"median": 0.15, "error plus": 0.46, "error minus": 0.45, "5th percentile": {-}0.3, "95th percentile": 0.61, "10th percentile": {-}0.2, "90th percentile": 0.51\}, "negfrac": \{"median": 0.44, "error plus": 0.12, "error minus": 0.13, "5th percentile": 0.31, "95th percentile": 0.57, "10th percentile": 0.34, "90th percentile": 0.54\}\}, "Continuum": \{"1percentile": \{"median": {-}0.98, "error plus": 0.036, "error minus": 0.013, "5th percentile": {-}0.99, "95th percentile": {-}0.94, "10th percentile": {-}0.99, "90th percentile": {-}0.95\}, "99percentile": \{"median": 0.98, "error plus": 0.013, "error minus": 0.031, "5th percentile": 0.94, "95th percentile": 0.99, "10th percentile": 0.95, "90th percentile": 0.99\}, "max": \{"median": {-}0.081, "error plus": 1.1, "error minus": 0.92, "5th percentile": {-}1.0, "95th percentile": 1.0, "10th percentile": {-}0.94, "90th percentile": 0.83\}, "log10gammafrac": \{"median": {-}0.025, "error plus": 0.54, "error minus": 0.52, "5th percentile": {-}0.54, "95th percentile": 0.52, "10th percentile": {-}0.43, "90th percentile": 0.4\}, "negfrac": \{"median": 0.5, "error plus": 0.19, "error minus": 0.19, "5th percentile": 0.31, "95th percentile": 0.69, "10th percentile": 0.35, "90th percentile": 0.65\}\}\}}\else\ifnum\pdfstrcmp{#1}{LowMassPeak}=0\let\macros@III@out\macros@XII\else\ifnum\pdfstrcmp{#1}{HighMassPeak}=0\let\macros@III@out\macros@XIII\else\ifnum\pdfstrcmp{#1}{Continuum}=0\let\macros@III@out\macros@XIV\else\def\macros@III@out{??}\fi\fi\fi\fi\macros@III@out}\newcommand\macros@IV[1][all]{\ifnum\pdfstrcmp{#1}{all}=0\def\macros@IV@out{\{"LowMassPeak": \{"Frac": \{"median": 0.77, "error plus": 0.11, "error minus": 0.17\}, "Percent": \{"median": 77, "error plus": 11, "error minus": 17\}\}, "HighMassPeak": \{"Frac": \{"median": 0.21, "error plus": 0.16, "error minus": 0.1\}, "Percent": \{"median": 21, "error plus": 16, "error minus": 10\}\}, "Continuum": \{"Frac": \{"median": 0.011, "error plus": 0.058, "error minus": 0.0096\}, "Percent": \{"median": 1.1, "error plus": 5.8, "error minus": 0.96\}\}\}}\else\ifnum\pdfstrcmp{#1}{LowMassPeak}=0\let\macros@IV@out\macros@XV\else\ifnum\pdfstrcmp{#1}{HighMassPeak}=0\let\macros@IV@out\macros@XVI\else\ifnum\pdfstrcmp{#1}{Continuum}=0\let\macros@IV@out\macros@XVII\else\def\macros@IV@out{??}\fi\fi\fi\fi\macros@IV@out}\newcommand\macros@V[1][all]{\ifnum\pdfstrcmp{#1}{all}=0\def\macros@V@out{\{"LowMassPeak": \{"median": 15, "error plus": 0.0, "error minus": 2.0\}, "HighMassPeak": \{"median": 50, "error plus": 4.0, "error minus": 13\}, "Continuum": \{"median": 5.0, "error plus": 13, "error minus": 4.0\}\}}\else\ifnum\pdfstrcmp{#1}{LowMassPeak}=0\let\macros@V@out\macros@XVIII\else\ifnum\pdfstrcmp{#1}{HighMassPeak}=0\let\macros@V@out\macros@XIX\else\ifnum\pdfstrcmp{#1}{Continuum}=0\let\macros@V@out\macros@XX\else\def\macros@V@out{??}\fi\fi\fi\fi\macros@V@out}\newcommand\macros@VI[1][all]{\ifnum\pdfstrcmp{#1}{all}=0\def\macros@VI@out{\{"1percentile": \{"median": 9.5, "error plus": 1.1, "error minus": 1.8, "5th percentile": 7.7, "95th percentile": 11, "10th percentile": 8.2, "90th percentile": 10\}, "99percentile": \{"median": 12, "error plus": 1.3, "error minus": 0.95, "5th percentile": 11, "95th percentile": 13, "10th percentile": 11, "90th percentile": 13\}, "max": \{"median": 11, "error plus": 0.83, "error minus": 0.95, "5th percentile": 9.8, "95th percentile": 12, "10th percentile": 10.0, "90th percentile": 11\}\}}\else\ifnum\pdfstrcmp{#1}{1percentile}=0\let\macros@VI@out\macros@XXI\else\ifnum\pdfstrcmp{#1}{99percentile}=0\let\macros@VI@out\macros@XXII\else\ifnum\pdfstrcmp{#1}{max}=0\let\macros@VI@out\macros@XXIII\else\def\macros@VI@out{??}\fi\fi\fi\fi\macros@VI@out}\newcommand\macros@VII[1][all]{\ifnum\pdfstrcmp{#1}{all}=0\def\macros@VII@out{\{"1percentile": \{"median": 6.0, "error plus": 5.6, "error minus": 0.71, "5th percentile": 5.2, "95th percentile": 12, "10th percentile": 5.4, "90th percentile": 9.6\}, "99percentile": \{"median": 50, "error plus": 7.1, "error minus": 6.7, "5th percentile": 44, "95th percentile": 57, "10th percentile": 45, "90th percentile": 56\}, "max": \{"median": 24, "error plus": 5.6, "error minus": 9.2, "5th percentile": 15, "95th percentile": 29, "10th percentile": 16, "90th percentile": 29\}\}}\else\ifnum\pdfstrcmp{#1}{1percentile}=0\let\macros@VII@out\macros@XXIV\else\ifnum\pdfstrcmp{#1}{99percentile}=0\let\macros@VII@out\macros@XXV\else\ifnum\pdfstrcmp{#1}{max}=0\let\macros@VII@out\macros@XXVI\else\def\macros@VII@out{??}\fi\fi\fi\fi\macros@VII@out}\newcommand\macros@VIII[1][all]{\ifnum\pdfstrcmp{#1}{all}=0\def\macros@VIII@out{\{"1percentile": \{"median": 11, "error plus": 16, "error minus": 4.8, "5th percentile": 6.3, "95th percentile": 27, "10th percentile": 6.8, "90th percentile": 22\}, "99percentile": \{"median": 90, "error plus": 9.5, "error minus": 26, "5th percentile": 64, "95th percentile": 100, "10th percentile": 73, "90th percentile": 99\}, "max": \{"median": 27, "error plus": 53, "error minus": 19, "5th percentile": 8.4, "95th percentile": 80, "10th percentile": 9.5, "90th percentile": 69\}\}}\else\ifnum\pdfstrcmp{#1}{1percentile}=0\let\macros@VIII@out\macros@XXVII\else\ifnum\pdfstrcmp{#1}{99percentile}=0\let\macros@VIII@out\macros@XXVIII\else\ifnum\pdfstrcmp{#1}{max}=0\let\macros@VIII@out\macros@XXIX\else\def\macros@VIII@out{??}\fi\fi\fi\fi\macros@VIII@out}\newcommand\macros@IX[1][all]{\ifnum\pdfstrcmp{#1}{all}=0\def\macros@IX@out{\{"1percentile": \{"median": 0.0067, "error plus": 0.01, "error minus": 0.0033, "5th percentile": 0.0033, "95th percentile": 0.017, "10th percentile": 0.0033, "90th percentile": 0.013\}, "99percentile": \{"median": 0.98, "error plus": 0.011, "error minus": 0.031, "5th percentile": 0.95, "95th percentile": 0.99, "10th percentile": 0.96, "90th percentile": 0.99\}, "max": \{"median": 0.16, "error plus": 0.19, "error minus": 0.16, "5th percentile": 0.0, "95th percentile": 0.35, "10th percentile": 0.023, "90th percentile": 0.3\}\}}\else\ifnum\pdfstrcmp{#1}{1percentile}=0\let\macros@IX@out\macros@XXX\else\ifnum\pdfstrcmp{#1}{99percentile}=0\let\macros@IX@out\macros@XXXI\else\ifnum\pdfstrcmp{#1}{max}=0\let\macros@IX@out\macros@XXXII\else\def\macros@IX@out{??}\fi\fi\fi\fi\macros@IX@out}\newcommand\macros@X[1][all]{\ifnum\pdfstrcmp{#1}{all}=0\def\macros@X@out{\{"1percentile": \{"median": 0.0078, "error plus": 0.013, "error minus": 0.0044, "5th percentile": 0.0033, "95th percentile": 0.021, "10th percentile": 0.0044, "90th percentile": 0.017\}, "99percentile": \{"median": 0.98, "error plus": 0.01, "error minus": 0.024, "5th percentile": 0.96, "95th percentile": 0.99, "10th percentile": 0.96, "90th percentile": 0.99\}, "max": \{"median": 0.23, "error plus": 0.28, "error minus": 0.23, "5th percentile": 0.0, "95th percentile": 0.51, "10th percentile": 0.0033, "90th percentile": 0.45\}\}}\else\ifnum\pdfstrcmp{#1}{1percentile}=0\let\macros@X@out\macros@XXXIII\else\ifnum\pdfstrcmp{#1}{99percentile}=0\let\macros@X@out\macros@XXXIV\else\ifnum\pdfstrcmp{#1}{max}=0\let\macros@X@out\macros@XXXV\else\def\macros@X@out{??}\fi\fi\fi\fi\macros@X@out}\newcommand\macros@XI[1][all]{\ifnum\pdfstrcmp{#1}{all}=0\def\macros@XI@out{\{"1percentile": \{"median": 0.011, "error plus": 0.018, "error minus": 0.0067, "5th percentile": 0.0044, "95th percentile": 0.029, "10th percentile": 0.0044, "90th percentile": 0.023\}, "99percentile": \{"median": 0.99, "error plus": 0.0067, "error minus": 0.017, "5th percentile": 0.97, "95th percentile": 0.99, "10th percentile": 0.98, "90th percentile": 0.99\}, "max": \{"median": 0.49, "error plus": 0.51, "error minus": 0.49, "5th percentile": 0.0, "95th percentile": 1.0, "10th percentile": 0.011, "90th percentile": 0.94\}\}}\else\ifnum\pdfstrcmp{#1}{1percentile}=0\let\macros@XI@out\macros@XXXVI\else\ifnum\pdfstrcmp{#1}{99percentile}=0\let\macros@XI@out\macros@XXXVII\else\ifnum\pdfstrcmp{#1}{max}=0\let\macros@XI@out\macros@XXXVIII\else\def\macros@XI@out{??}\fi\fi\fi\fi\macros@XI@out}\newcommand\macros@XII[1][all]{\ifnum\pdfstrcmp{#1}{all}=0\def\macros@XII@out{\{"1percentile": \{"median": {-}0.97, "error plus": 0.047, "error minus": 0.02, "5th percentile": {-}0.99, "95th percentile": {-}0.92, "10th percentile": {-}0.98, "90th percentile": {-}0.93\}, "99percentile": \{"median": 0.98, "error plus": 0.013, "error minus": 0.031, "5th percentile": 0.95, "95th percentile": 0.99, "10th percentile": 0.96, "90th percentile": 0.99\}, "max": \{"median": 0.49, "error plus": 0.51, "error minus": 0.82, "5th percentile": {-}0.33, "95th percentile": 1.0, "10th percentile": {-}0.0056, "90th percentile": 0.9\}, "log10gammafrac": \{"median": 0.24, "error plus": 0.5, "error minus": 0.5, "5th percentile": {-}0.26, "95th percentile": 0.74, "10th percentile": {-}0.15, "90th percentile": 0.62\}, "negfrac": \{"median": 0.35, "error plus": 0.14, "error minus": 0.12, "5th percentile": 0.22, "95th percentile": 0.49, "10th percentile": 0.25, "90th percentile": 0.46\}\}}\else\ifnum\pdfstrcmp{#1}{1percentile}=0\let\macros@XII@out\macros@XXXIX\else\ifnum\pdfstrcmp{#1}{99percentile}=0\let\macros@XII@out\macros@XL\else\ifnum\pdfstrcmp{#1}{max}=0\let\macros@XII@out\macros@XLI\else\ifnum\pdfstrcmp{#1}{log10gammafrac}=0\let\macros@XII@out\macros@XLII\else\ifnum\pdfstrcmp{#1}{negfrac}=0\let\macros@XII@out\macros@XLIII\else\def\macros@XII@out{??}\fi\fi\fi\fi\fi\fi\macros@XII@out}\newcommand\macros@XIII[1][all]{\ifnum\pdfstrcmp{#1}{all}=0\def\macros@XIII@out{\{"1percentile": \{"median": {-}0.97, "error plus": 0.044, "error minus": 0.016, "5th percentile": {-}0.99, "95th percentile": {-}0.93, "10th percentile": {-}0.98, "90th percentile": {-}0.94\}, "99percentile": \{"median": 0.98, "error plus": 0.011, "error minus": 0.027, "5th percentile": 0.95, "95th percentile": 0.99, "10th percentile": 0.96, "90th percentile": 0.99\}, "max": \{"median": 0.2, "error plus": 0.8, "error minus": 0.77, "5th percentile": {-}0.58, "95th percentile": 1.0, "10th percentile": {-}0.34, "90th percentile": 0.84\}, "log10gammafrac": \{"median": 0.15, "error plus": 0.46, "error minus": 0.45, "5th percentile": {-}0.3, "95th percentile": 0.61, "10th percentile": {-}0.2, "90th percentile": 0.51\}, "negfrac": \{"median": 0.44, "error plus": 0.12, "error minus": 0.13, "5th percentile": 0.31, "95th percentile": 0.57, "10th percentile": 0.34, "90th percentile": 0.54\}\}}\else\ifnum\pdfstrcmp{#1}{1percentile}=0\let\macros@XIII@out\macros@XLIV\else\ifnum\pdfstrcmp{#1}{99percentile}=0\let\macros@XIII@out\macros@XLV\else\ifnum\pdfstrcmp{#1}{max}=0\let\macros@XIII@out\macros@XLVI\else\ifnum\pdfstrcmp{#1}{log10gammafrac}=0\let\macros@XIII@out\macros@XLVII\else\ifnum\pdfstrcmp{#1}{negfrac}=0\let\macros@XIII@out\macros@XLVIII\else\def\macros@XIII@out{??}\fi\fi\fi\fi\fi\fi\macros@XIII@out}\newcommand\macros@XIV[1][all]{\ifnum\pdfstrcmp{#1}{all}=0\def\macros@XIV@out{\{"1percentile": \{"median": {-}0.98, "error plus": 0.036, "error minus": 0.013, "5th percentile": {-}0.99, "95th percentile": {-}0.94, "10th percentile": {-}0.99, "90th percentile": {-}0.95\}, "99percentile": \{"median": 0.98, "error plus": 0.013, "error minus": 0.031, "5th percentile": 0.94, "95th percentile": 0.99, "10th percentile": 0.95, "90th percentile": 0.99\}, "max": \{"median": {-}0.081, "error plus": 1.1, "error minus": 0.92, "5th percentile": {-}1.0, "95th percentile": 1.0, "10th percentile": {-}0.94, "90th percentile": 0.83\}, "log10gammafrac": \{"median": {-}0.025, "error plus": 0.54, "error minus": 0.52, "5th percentile": {-}0.54, "95th percentile": 0.52, "10th percentile": {-}0.43, "90th percentile": 0.4\}, "negfrac": \{"median": 0.5, "error plus": 0.19, "error minus": 0.19, "5th percentile": 0.31, "95th percentile": 0.69, "10th percentile": 0.35, "90th percentile": 0.65\}\}}\else\ifnum\pdfstrcmp{#1}{1percentile}=0\let\macros@XIV@out\macros@XLIX\else\ifnum\pdfstrcmp{#1}{99percentile}=0\let\macros@XIV@out\macros@L\else\ifnum\pdfstrcmp{#1}{max}=0\let\macros@XIV@out\macros@LI\else\ifnum\pdfstrcmp{#1}{log10gammafrac}=0\let\macros@XIV@out\macros@LII\else\ifnum\pdfstrcmp{#1}{negfrac}=0\let\macros@XIV@out\macros@LIII\else\def\macros@XIV@out{??}\fi\fi\fi\fi\fi\fi\macros@XIV@out}\newcommand\macros@XV[1][all]{\ifnum\pdfstrcmp{#1}{all}=0\def\macros@XV@out{\{"Frac": \{"median": 0.77, "error plus": 0.11, "error minus": 0.17\}, "Percent": \{"median": 77, "error plus": 11, "error minus": 17\}\}}\else\ifnum\pdfstrcmp{#1}{Frac}=0\let\macros@XV@out\macros@LIV\else\ifnum\pdfstrcmp{#1}{Percent}=0\let\macros@XV@out\macros@LV\else\def\macros@XV@out{??}\fi\fi\fi\macros@XV@out}\newcommand\macros@XVI[1][all]{\ifnum\pdfstrcmp{#1}{all}=0\def\macros@XVI@out{\{"Frac": \{"median": 0.21, "error plus": 0.16, "error minus": 0.1\}, "Percent": \{"median": 21, "error plus": 16, "error minus": 10\}\}}\else\ifnum\pdfstrcmp{#1}{Frac}=0\let\macros@XVI@out\macros@LVI\else\ifnum\pdfstrcmp{#1}{Percent}=0\let\macros@XVI@out\macros@LVII\else\def\macros@XVI@out{??}\fi\fi\fi\macros@XVI@out}\newcommand\macros@XVII[1][all]{\ifnum\pdfstrcmp{#1}{all}=0\def\macros@XVII@out{\{"Frac": \{"median": 0.011, "error plus": 0.058, "error minus": 0.0096\}, "Percent": \{"median": 1.1, "error plus": 5.8, "error minus": 0.96\}\}}\else\ifnum\pdfstrcmp{#1}{Frac}=0\let\macros@XVII@out\macros@LVIII\else\ifnum\pdfstrcmp{#1}{Percent}=0\let\macros@XVII@out\macros@LIX\else\def\macros@XVII@out{??}\fi\fi\fi\macros@XVII@out}\newcommand\macros@XVIII[1][all]{\ifnum\pdfstrcmp{#1}{all}=0\def\macros@XVIII@out{\{"median": 15, "error plus": 0.0, "error minus": 2.0\}}\else\ifnum\pdfstrcmp{#1}{median}=0\def\macros@XVIII@out{15}\else\ifnum\pdfstrcmp{#1}{error plus}=0\def\macros@XVIII@out{0.0}\else\ifnum\pdfstrcmp{#1}{error minus}=0\def\macros@XVIII@out{2.0}\else\def\macros@XVIII@out{??}\fi\fi\fi\fi\macros@XVIII@out}\newcommand\macros@XIX[1][all]{\ifnum\pdfstrcmp{#1}{all}=0\def\macros@XIX@out{\{"median": 50, "error plus": 4.0, "error minus": 13\}}\else\ifnum\pdfstrcmp{#1}{median}=0\def\macros@XIX@out{50}\else\ifnum\pdfstrcmp{#1}{error plus}=0\def\macros@XIX@out{4.0}\else\ifnum\pdfstrcmp{#1}{error minus}=0\def\macros@XIX@out{13}\else\def\macros@XIX@out{??}\fi\fi\fi\fi\macros@XIX@out}\newcommand\macros@XX[1][all]{\ifnum\pdfstrcmp{#1}{all}=0\def\macros@XX@out{\{"median": 5.0, "error plus": 13, "error minus": 4.0\}}\else\ifnum\pdfstrcmp{#1}{median}=0\def\macros@XX@out{5.0}\else\ifnum\pdfstrcmp{#1}{error plus}=0\def\macros@XX@out{13}\else\ifnum\pdfstrcmp{#1}{error minus}=0\def\macros@XX@out{4.0}\else\def\macros@XX@out{??}\fi\fi\fi\fi\macros@XX@out}\newcommand\macros@XXI[1][all]{\ifnum\pdfstrcmp{#1}{all}=0\def\macros@XXI@out{\{"median": 9.5, "error plus": 1.1, "error minus": 1.8, "5th percentile": 7.7, "95th percentile": 11, "10th percentile": 8.2, "90th percentile": 10\}}\else\ifnum\pdfstrcmp{#1}{median}=0\def\macros@XXI@out{9.5}\else\ifnum\pdfstrcmp{#1}{error plus}=0\def\macros@XXI@out{1.1}\else\ifnum\pdfstrcmp{#1}{error minus}=0\def\macros@XXI@out{1.8}\else\ifnum\pdfstrcmp{#1}{5th percentile}=0\def\macros@XXI@out{7.7}\else\ifnum\pdfstrcmp{#1}{95th percentile}=0\def\macros@XXI@out{11}\else\ifnum\pdfstrcmp{#1}{10th percentile}=0\def\macros@XXI@out{8.2}\else\ifnum\pdfstrcmp{#1}{90th percentile}=0\def\macros@XXI@out{10}\else\def\macros@XXI@out{??}\fi\fi\fi\fi\fi\fi\fi\fi\macros@XXI@out}\newcommand\macros@XXII[1][all]{\ifnum\pdfstrcmp{#1}{all}=0\def\macros@XXII@out{\{"median": 12, "error plus": 1.3, "error minus": 0.95, "5th percentile": 11, "95th percentile": 13, "10th percentile": 11, "90th percentile": 13\}}\else\ifnum\pdfstrcmp{#1}{median}=0\def\macros@XXII@out{12}\else\ifnum\pdfstrcmp{#1}{error plus}=0\def\macros@XXII@out{1.3}\else\ifnum\pdfstrcmp{#1}{error minus}=0\def\macros@XXII@out{0.95}\else\ifnum\pdfstrcmp{#1}{5th percentile}=0\def\macros@XXII@out{11}\else\ifnum\pdfstrcmp{#1}{95th percentile}=0\def\macros@XXII@out{13}\else\ifnum\pdfstrcmp{#1}{10th percentile}=0\def\macros@XXII@out{11}\else\ifnum\pdfstrcmp{#1}{90th percentile}=0\def\macros@XXII@out{13}\else\def\macros@XXII@out{??}\fi\fi\fi\fi\fi\fi\fi\fi\macros@XXII@out}\newcommand\macros@XXIII[1][all]{\ifnum\pdfstrcmp{#1}{all}=0\def\macros@XXIII@out{\{"median": 11, "error plus": 0.83, "error minus": 0.95, "5th percentile": 9.8, "95th percentile": 12, "10th percentile": 10.0, "90th percentile": 11\}}\else\ifnum\pdfstrcmp{#1}{median}=0\def\macros@XXIII@out{11}\else\ifnum\pdfstrcmp{#1}{error plus}=0\def\macros@XXIII@out{0.83}\else\ifnum\pdfstrcmp{#1}{error minus}=0\def\macros@XXIII@out{0.95}\else\ifnum\pdfstrcmp{#1}{5th percentile}=0\def\macros@XXIII@out{9.8}\else\ifnum\pdfstrcmp{#1}{95th percentile}=0\def\macros@XXIII@out{12}\else\ifnum\pdfstrcmp{#1}{10th percentile}=0\def\macros@XXIII@out{10.0}\else\ifnum\pdfstrcmp{#1}{90th percentile}=0\def\macros@XXIII@out{11}\else\def\macros@XXIII@out{??}\fi\fi\fi\fi\fi\fi\fi\fi\macros@XXIII@out}\newcommand\macros@XXIV[1][all]{\ifnum\pdfstrcmp{#1}{all}=0\def\macros@XXIV@out{\{"median": 6.0, "error plus": 5.6, "error minus": 0.71, "5th percentile": 5.2, "95th percentile": 12, "10th percentile": 5.4, "90th percentile": 9.6\}}\else\ifnum\pdfstrcmp{#1}{median}=0\def\macros@XXIV@out{6.0}\else\ifnum\pdfstrcmp{#1}{error plus}=0\def\macros@XXIV@out{5.6}\else\ifnum\pdfstrcmp{#1}{error minus}=0\def\macros@XXIV@out{0.71}\else\ifnum\pdfstrcmp{#1}{5th percentile}=0\def\macros@XXIV@out{5.2}\else\ifnum\pdfstrcmp{#1}{95th percentile}=0\def\macros@XXIV@out{12}\else\ifnum\pdfstrcmp{#1}{10th percentile}=0\def\macros@XXIV@out{5.4}\else\ifnum\pdfstrcmp{#1}{90th percentile}=0\def\macros@XXIV@out{9.6}\else\def\macros@XXIV@out{??}\fi\fi\fi\fi\fi\fi\fi\fi\macros@XXIV@out}\newcommand\macros@XXV[1][all]{\ifnum\pdfstrcmp{#1}{all}=0\def\macros@XXV@out{\{"median": 50, "error plus": 7.1, "error minus": 6.7, "5th percentile": 44, "95th percentile": 57, "10th percentile": 45, "90th percentile": 56\}}\else\ifnum\pdfstrcmp{#1}{median}=0\def\macros@XXV@out{50}\else\ifnum\pdfstrcmp{#1}{error plus}=0\def\macros@XXV@out{7.1}\else\ifnum\pdfstrcmp{#1}{error minus}=0\def\macros@XXV@out{6.7}\else\ifnum\pdfstrcmp{#1}{5th percentile}=0\def\macros@XXV@out{44}\else\ifnum\pdfstrcmp{#1}{95th percentile}=0\def\macros@XXV@out{57}\else\ifnum\pdfstrcmp{#1}{10th percentile}=0\def\macros@XXV@out{45}\else\ifnum\pdfstrcmp{#1}{90th percentile}=0\def\macros@XXV@out{56}\else\def\macros@XXV@out{??}\fi\fi\fi\fi\fi\fi\fi\fi\macros@XXV@out}\newcommand\macros@XXVI[1][all]{\ifnum\pdfstrcmp{#1}{all}=0\def\macros@XXVI@out{\{"median": 24, "error plus": 5.6, "error minus": 9.2, "5th percentile": 15, "95th percentile": 29, "10th percentile": 16, "90th percentile": 29\}}\else\ifnum\pdfstrcmp{#1}{median}=0\def\macros@XXVI@out{24}\else\ifnum\pdfstrcmp{#1}{error plus}=0\def\macros@XXVI@out{5.6}\else\ifnum\pdfstrcmp{#1}{error minus}=0\def\macros@XXVI@out{9.2}\else\ifnum\pdfstrcmp{#1}{5th percentile}=0\def\macros@XXVI@out{15}\else\ifnum\pdfstrcmp{#1}{95th percentile}=0\def\macros@XXVI@out{29}\else\ifnum\pdfstrcmp{#1}{10th percentile}=0\def\macros@XXVI@out{16}\else\ifnum\pdfstrcmp{#1}{90th percentile}=0\def\macros@XXVI@out{29}\else\def\macros@XXVI@out{??}\fi\fi\fi\fi\fi\fi\fi\fi\macros@XXVI@out}\newcommand\macros@XXVII[1][all]{\ifnum\pdfstrcmp{#1}{all}=0\def\macros@XXVII@out{\{"median": 11, "error plus": 16, "error minus": 4.8, "5th percentile": 6.3, "95th percentile": 27, "10th percentile": 6.8, "90th percentile": 22\}}\else\ifnum\pdfstrcmp{#1}{median}=0\def\macros@XXVII@out{11}\else\ifnum\pdfstrcmp{#1}{error plus}=0\def\macros@XXVII@out{16}\else\ifnum\pdfstrcmp{#1}{error minus}=0\def\macros@XXVII@out{4.8}\else\ifnum\pdfstrcmp{#1}{5th percentile}=0\def\macros@XXVII@out{6.3}\else\ifnum\pdfstrcmp{#1}{95th percentile}=0\def\macros@XXVII@out{27}\else\ifnum\pdfstrcmp{#1}{10th percentile}=0\def\macros@XXVII@out{6.8}\else\ifnum\pdfstrcmp{#1}{90th percentile}=0\def\macros@XXVII@out{22}\else\def\macros@XXVII@out{??}\fi\fi\fi\fi\fi\fi\fi\fi\macros@XXVII@out}\newcommand\macros@XXVIII[1][all]{\ifnum\pdfstrcmp{#1}{all}=0\def\macros@XXVIII@out{\{"median": 90, "error plus": 9.5, "error minus": 26, "5th percentile": 64, "95th percentile": 100, "10th percentile": 73, "90th percentile": 99\}}\else\ifnum\pdfstrcmp{#1}{median}=0\def\macros@XXVIII@out{90}\else\ifnum\pdfstrcmp{#1}{error plus}=0\def\macros@XXVIII@out{9.5}\else\ifnum\pdfstrcmp{#1}{error minus}=0\def\macros@XXVIII@out{26}\else\ifnum\pdfstrcmp{#1}{5th percentile}=0\def\macros@XXVIII@out{64}\else\ifnum\pdfstrcmp{#1}{95th percentile}=0\def\macros@XXVIII@out{100}\else\ifnum\pdfstrcmp{#1}{10th percentile}=0\def\macros@XXVIII@out{73}\else\ifnum\pdfstrcmp{#1}{90th percentile}=0\def\macros@XXVIII@out{99}\else\def\macros@XXVIII@out{??}\fi\fi\fi\fi\fi\fi\fi\fi\macros@XXVIII@out}\newcommand\macros@XXIX[1][all]{\ifnum\pdfstrcmp{#1}{all}=0\def\macros@XXIX@out{\{"median": 27, "error plus": 53, "error minus": 19, "5th percentile": 8.4, "95th percentile": 80, "10th percentile": 9.5, "90th percentile": 69\}}\else\ifnum\pdfstrcmp{#1}{median}=0\def\macros@XXIX@out{27}\else\ifnum\pdfstrcmp{#1}{error plus}=0\def\macros@XXIX@out{53}\else\ifnum\pdfstrcmp{#1}{error minus}=0\def\macros@XXIX@out{19}\else\ifnum\pdfstrcmp{#1}{5th percentile}=0\def\macros@XXIX@out{8.4}\else\ifnum\pdfstrcmp{#1}{95th percentile}=0\def\macros@XXIX@out{80}\else\ifnum\pdfstrcmp{#1}{10th percentile}=0\def\macros@XXIX@out{9.5}\else\ifnum\pdfstrcmp{#1}{90th percentile}=0\def\macros@XXIX@out{69}\else\def\macros@XXIX@out{??}\fi\fi\fi\fi\fi\fi\fi\fi\macros@XXIX@out}\newcommand\macros@XXX[1][all]{\ifnum\pdfstrcmp{#1}{all}=0\def\macros@XXX@out{\{"median": 0.0067, "error plus": 0.01, "error minus": 0.0033, "5th percentile": 0.0033, "95th percentile": 0.017, "10th percentile": 0.0033, "90th percentile": 0.013\}}\else\ifnum\pdfstrcmp{#1}{median}=0\def\macros@XXX@out{0.0067}\else\ifnum\pdfstrcmp{#1}{error plus}=0\def\macros@XXX@out{0.01}\else\ifnum\pdfstrcmp{#1}{error minus}=0\def\macros@XXX@out{0.0033}\else\ifnum\pdfstrcmp{#1}{5th percentile}=0\def\macros@XXX@out{0.0033}\else\ifnum\pdfstrcmp{#1}{95th percentile}=0\def\macros@XXX@out{0.017}\else\ifnum\pdfstrcmp{#1}{10th percentile}=0\def\macros@XXX@out{0.0033}\else\ifnum\pdfstrcmp{#1}{90th percentile}=0\def\macros@XXX@out{0.013}\else\def\macros@XXX@out{??}\fi\fi\fi\fi\fi\fi\fi\fi\macros@XXX@out}\newcommand\macros@XXXI[1][all]{\ifnum\pdfstrcmp{#1}{all}=0\def\macros@XXXI@out{\{"median": 0.98, "error plus": 0.011, "error minus": 0.031, "5th percentile": 0.95, "95th percentile": 0.99, "10th percentile": 0.96, "90th percentile": 0.99\}}\else\ifnum\pdfstrcmp{#1}{median}=0\def\macros@XXXI@out{0.98}\else\ifnum\pdfstrcmp{#1}{error plus}=0\def\macros@XXXI@out{0.011}\else\ifnum\pdfstrcmp{#1}{error minus}=0\def\macros@XXXI@out{0.031}\else\ifnum\pdfstrcmp{#1}{5th percentile}=0\def\macros@XXXI@out{0.95}\else\ifnum\pdfstrcmp{#1}{95th percentile}=0\def\macros@XXXI@out{0.99}\else\ifnum\pdfstrcmp{#1}{10th percentile}=0\def\macros@XXXI@out{0.96}\else\ifnum\pdfstrcmp{#1}{90th percentile}=0\def\macros@XXXI@out{0.99}\else\def\macros@XXXI@out{??}\fi\fi\fi\fi\fi\fi\fi\fi\macros@XXXI@out}\newcommand\macros@XXXII[1][all]{\ifnum\pdfstrcmp{#1}{all}=0\def\macros@XXXII@out{\{"median": 0.16, "error plus": 0.19, "error minus": 0.16, "5th percentile": 0.0, "95th percentile": 0.35, "10th percentile": 0.023, "90th percentile": 0.3\}}\else\ifnum\pdfstrcmp{#1}{median}=0\def\macros@XXXII@out{0.16}\else\ifnum\pdfstrcmp{#1}{error plus}=0\def\macros@XXXII@out{0.19}\else\ifnum\pdfstrcmp{#1}{error minus}=0\def\macros@XXXII@out{0.16}\else\ifnum\pdfstrcmp{#1}{5th percentile}=0\def\macros@XXXII@out{0.0}\else\ifnum\pdfstrcmp{#1}{95th percentile}=0\def\macros@XXXII@out{0.35}\else\ifnum\pdfstrcmp{#1}{10th percentile}=0\def\macros@XXXII@out{0.023}\else\ifnum\pdfstrcmp{#1}{90th percentile}=0\def\macros@XXXII@out{0.3}\else\def\macros@XXXII@out{??}\fi\fi\fi\fi\fi\fi\fi\fi\macros@XXXII@out}\newcommand\macros@XXXIII[1][all]{\ifnum\pdfstrcmp{#1}{all}=0\def\macros@XXXIII@out{\{"median": 0.0078, "error plus": 0.013, "error minus": 0.0044, "5th percentile": 0.0033, "95th percentile": 0.021, "10th percentile": 0.0044, "90th percentile": 0.017\}}\else\ifnum\pdfstrcmp{#1}{median}=0\def\macros@XXXIII@out{0.0078}\else\ifnum\pdfstrcmp{#1}{error plus}=0\def\macros@XXXIII@out{0.013}\else\ifnum\pdfstrcmp{#1}{error minus}=0\def\macros@XXXIII@out{0.0044}\else\ifnum\pdfstrcmp{#1}{5th percentile}=0\def\macros@XXXIII@out{0.0033}\else\ifnum\pdfstrcmp{#1}{95th percentile}=0\def\macros@XXXIII@out{0.021}\else\ifnum\pdfstrcmp{#1}{10th percentile}=0\def\macros@XXXIII@out{0.0044}\else\ifnum\pdfstrcmp{#1}{90th percentile}=0\def\macros@XXXIII@out{0.017}\else\def\macros@XXXIII@out{??}\fi\fi\fi\fi\fi\fi\fi\fi\macros@XXXIII@out}\newcommand\macros@XXXIV[1][all]{\ifnum\pdfstrcmp{#1}{all}=0\def\macros@XXXIV@out{\{"median": 0.98, "error plus": 0.01, "error minus": 0.024, "5th percentile": 0.96, "95th percentile": 0.99, "10th percentile": 0.96, "90th percentile": 0.99\}}\else\ifnum\pdfstrcmp{#1}{median}=0\def\macros@XXXIV@out{0.98}\else\ifnum\pdfstrcmp{#1}{error plus}=0\def\macros@XXXIV@out{0.01}\else\ifnum\pdfstrcmp{#1}{error minus}=0\def\macros@XXXIV@out{0.024}\else\ifnum\pdfstrcmp{#1}{5th percentile}=0\def\macros@XXXIV@out{0.96}\else\ifnum\pdfstrcmp{#1}{95th percentile}=0\def\macros@XXXIV@out{0.99}\else\ifnum\pdfstrcmp{#1}{10th percentile}=0\def\macros@XXXIV@out{0.96}\else\ifnum\pdfstrcmp{#1}{90th percentile}=0\def\macros@XXXIV@out{0.99}\else\def\macros@XXXIV@out{??}\fi\fi\fi\fi\fi\fi\fi\fi\macros@XXXIV@out}\newcommand\macros@XXXV[1][all]{\ifnum\pdfstrcmp{#1}{all}=0\def\macros@XXXV@out{\{"median": 0.23, "error plus": 0.28, "error minus": 0.23, "5th percentile": 0.0, "95th percentile": 0.51, "10th percentile": 0.0033, "90th percentile": 0.45\}}\else\ifnum\pdfstrcmp{#1}{median}=0\def\macros@XXXV@out{0.23}\else\ifnum\pdfstrcmp{#1}{error plus}=0\def\macros@XXXV@out{0.28}\else\ifnum\pdfstrcmp{#1}{error minus}=0\def\macros@XXXV@out{0.23}\else\ifnum\pdfstrcmp{#1}{5th percentile}=0\def\macros@XXXV@out{0.0}\else\ifnum\pdfstrcmp{#1}{95th percentile}=0\def\macros@XXXV@out{0.51}\else\ifnum\pdfstrcmp{#1}{10th percentile}=0\def\macros@XXXV@out{0.0033}\else\ifnum\pdfstrcmp{#1}{90th percentile}=0\def\macros@XXXV@out{0.45}\else\def\macros@XXXV@out{??}\fi\fi\fi\fi\fi\fi\fi\fi\macros@XXXV@out}\newcommand\macros@XXXVI[1][all]{\ifnum\pdfstrcmp{#1}{all}=0\def\macros@XXXVI@out{\{"median": 0.011, "error plus": 0.018, "error minus": 0.0067, "5th percentile": 0.0044, "95th percentile": 0.029, "10th percentile": 0.0044, "90th percentile": 0.023\}}\else\ifnum\pdfstrcmp{#1}{median}=0\def\macros@XXXVI@out{0.011}\else\ifnum\pdfstrcmp{#1}{error plus}=0\def\macros@XXXVI@out{0.018}\else\ifnum\pdfstrcmp{#1}{error minus}=0\def\macros@XXXVI@out{0.0067}\else\ifnum\pdfstrcmp{#1}{5th percentile}=0\def\macros@XXXVI@out{0.0044}\else\ifnum\pdfstrcmp{#1}{95th percentile}=0\def\macros@XXXVI@out{0.029}\else\ifnum\pdfstrcmp{#1}{10th percentile}=0\def\macros@XXXVI@out{0.0044}\else\ifnum\pdfstrcmp{#1}{90th percentile}=0\def\macros@XXXVI@out{0.023}\else\def\macros@XXXVI@out{??}\fi\fi\fi\fi\fi\fi\fi\fi\macros@XXXVI@out}\newcommand\macros@XXXVII[1][all]{\ifnum\pdfstrcmp{#1}{all}=0\def\macros@XXXVII@out{\{"median": 0.99, "error plus": 0.0067, "error minus": 0.017, "5th percentile": 0.97, "95th percentile": 0.99, "10th percentile": 0.98, "90th percentile": 0.99\}}\else\ifnum\pdfstrcmp{#1}{median}=0\def\macros@XXXVII@out{0.99}\else\ifnum\pdfstrcmp{#1}{error plus}=0\def\macros@XXXVII@out{0.0067}\else\ifnum\pdfstrcmp{#1}{error minus}=0\def\macros@XXXVII@out{0.017}\else\ifnum\pdfstrcmp{#1}{5th percentile}=0\def\macros@XXXVII@out{0.97}\else\ifnum\pdfstrcmp{#1}{95th percentile}=0\def\macros@XXXVII@out{0.99}\else\ifnum\pdfstrcmp{#1}{10th percentile}=0\def\macros@XXXVII@out{0.98}\else\ifnum\pdfstrcmp{#1}{90th percentile}=0\def\macros@XXXVII@out{0.99}\else\def\macros@XXXVII@out{??}\fi\fi\fi\fi\fi\fi\fi\fi\macros@XXXVII@out}\newcommand\macros@XXXVIII[1][all]{\ifnum\pdfstrcmp{#1}{all}=0\def\macros@XXXVIII@out{\{"median": 0.49, "error plus": 0.51, "error minus": 0.49, "5th percentile": 0.0, "95th percentile": 1.0, "10th percentile": 0.011, "90th percentile": 0.94\}}\else\ifnum\pdfstrcmp{#1}{median}=0\def\macros@XXXVIII@out{0.49}\else\ifnum\pdfstrcmp{#1}{error plus}=0\def\macros@XXXVIII@out{0.51}\else\ifnum\pdfstrcmp{#1}{error minus}=0\def\macros@XXXVIII@out{0.49}\else\ifnum\pdfstrcmp{#1}{5th percentile}=0\def\macros@XXXVIII@out{0.0}\else\ifnum\pdfstrcmp{#1}{95th percentile}=0\def\macros@XXXVIII@out{1.0}\else\ifnum\pdfstrcmp{#1}{10th percentile}=0\def\macros@XXXVIII@out{0.011}\else\ifnum\pdfstrcmp{#1}{90th percentile}=0\def\macros@XXXVIII@out{0.94}\else\def\macros@XXXVIII@out{??}\fi\fi\fi\fi\fi\fi\fi\fi\macros@XXXVIII@out}\newcommand\macros@XXXIX[1][all]{\ifnum\pdfstrcmp{#1}{all}=0\def\macros@XXXIX@out{\{"median": {-}0.97, "error plus": 0.047, "error minus": 0.02, "5th percentile": {-}0.99, "95th percentile": {-}0.92, "10th percentile": {-}0.98, "90th percentile": {-}0.93\}}\else\ifnum\pdfstrcmp{#1}{median}=0\def\macros@XXXIX@out{{-}0.97}\else\ifnum\pdfstrcmp{#1}{error plus}=0\def\macros@XXXIX@out{0.047}\else\ifnum\pdfstrcmp{#1}{error minus}=0\def\macros@XXXIX@out{0.02}\else\ifnum\pdfstrcmp{#1}{5th percentile}=0\def\macros@XXXIX@out{{-}0.99}\else\ifnum\pdfstrcmp{#1}{95th percentile}=0\def\macros@XXXIX@out{{-}0.92}\else\ifnum\pdfstrcmp{#1}{10th percentile}=0\def\macros@XXXIX@out{{-}0.98}\else\ifnum\pdfstrcmp{#1}{90th percentile}=0\def\macros@XXXIX@out{{-}0.93}\else\def\macros@XXXIX@out{??}\fi\fi\fi\fi\fi\fi\fi\fi\macros@XXXIX@out}\newcommand\macros@XL[1][all]{\ifnum\pdfstrcmp{#1}{all}=0\def\macros@XL@out{\{"median": 0.98, "error plus": 0.013, "error minus": 0.031, "5th percentile": 0.95, "95th percentile": 0.99, "10th percentile": 0.96, "90th percentile": 0.99\}}\else\ifnum\pdfstrcmp{#1}{median}=0\def\macros@XL@out{0.98}\else\ifnum\pdfstrcmp{#1}{error plus}=0\def\macros@XL@out{0.013}\else\ifnum\pdfstrcmp{#1}{error minus}=0\def\macros@XL@out{0.031}\else\ifnum\pdfstrcmp{#1}{5th percentile}=0\def\macros@XL@out{0.95}\else\ifnum\pdfstrcmp{#1}{95th percentile}=0\def\macros@XL@out{0.99}\else\ifnum\pdfstrcmp{#1}{10th percentile}=0\def\macros@XL@out{0.96}\else\ifnum\pdfstrcmp{#1}{90th percentile}=0\def\macros@XL@out{0.99}\else\def\macros@XL@out{??}\fi\fi\fi\fi\fi\fi\fi\fi\macros@XL@out}\newcommand\macros@XLI[1][all]{\ifnum\pdfstrcmp{#1}{all}=0\def\macros@XLI@out{\{"median": 0.49, "error plus": 0.51, "error minus": 0.82, "5th percentile": {-}0.33, "95th percentile": 1.0, "10th percentile": {-}0.0056, "90th percentile": 0.9\}}\else\ifnum\pdfstrcmp{#1}{median}=0\def\macros@XLI@out{0.49}\else\ifnum\pdfstrcmp{#1}{error plus}=0\def\macros@XLI@out{0.51}\else\ifnum\pdfstrcmp{#1}{error minus}=0\def\macros@XLI@out{0.82}\else\ifnum\pdfstrcmp{#1}{5th percentile}=0\def\macros@XLI@out{{-}0.33}\else\ifnum\pdfstrcmp{#1}{95th percentile}=0\def\macros@XLI@out{1.0}\else\ifnum\pdfstrcmp{#1}{10th percentile}=0\def\macros@XLI@out{{-}0.0056}\else\ifnum\pdfstrcmp{#1}{90th percentile}=0\def\macros@XLI@out{0.9}\else\def\macros@XLI@out{??}\fi\fi\fi\fi\fi\fi\fi\fi\macros@XLI@out}\newcommand\macros@XLII[1][all]{\ifnum\pdfstrcmp{#1}{all}=0\def\macros@XLII@out{\{"median": 0.24, "error plus": 0.5, "error minus": 0.5, "5th percentile": {-}0.26, "95th percentile": 0.74, "10th percentile": {-}0.15, "90th percentile": 0.62\}}\else\ifnum\pdfstrcmp{#1}{median}=0\def\macros@XLII@out{0.24}\else\ifnum\pdfstrcmp{#1}{error plus}=0\def\macros@XLII@out{0.5}\else\ifnum\pdfstrcmp{#1}{error minus}=0\def\macros@XLII@out{0.5}\else\ifnum\pdfstrcmp{#1}{5th percentile}=0\def\macros@XLII@out{{-}0.26}\else\ifnum\pdfstrcmp{#1}{95th percentile}=0\def\macros@XLII@out{0.74}\else\ifnum\pdfstrcmp{#1}{10th percentile}=0\def\macros@XLII@out{{-}0.15}\else\ifnum\pdfstrcmp{#1}{90th percentile}=0\def\macros@XLII@out{0.62}\else\def\macros@XLII@out{??}\fi\fi\fi\fi\fi\fi\fi\fi\macros@XLII@out}\newcommand\macros@XLIII[1][all]{\ifnum\pdfstrcmp{#1}{all}=0\def\macros@XLIII@out{\{"median": 0.35, "error plus": 0.14, "error minus": 0.12, "5th percentile": 0.22, "95th percentile": 0.49, "10th percentile": 0.25, "90th percentile": 0.46\}}\else\ifnum\pdfstrcmp{#1}{median}=0\def\macros@XLIII@out{0.35}\else\ifnum\pdfstrcmp{#1}{error plus}=0\def\macros@XLIII@out{0.14}\else\ifnum\pdfstrcmp{#1}{error minus}=0\def\macros@XLIII@out{0.12}\else\ifnum\pdfstrcmp{#1}{5th percentile}=0\def\macros@XLIII@out{0.22}\else\ifnum\pdfstrcmp{#1}{95th percentile}=0\def\macros@XLIII@out{0.49}\else\ifnum\pdfstrcmp{#1}{10th percentile}=0\def\macros@XLIII@out{0.25}\else\ifnum\pdfstrcmp{#1}{90th percentile}=0\def\macros@XLIII@out{0.46}\else\def\macros@XLIII@out{??}\fi\fi\fi\fi\fi\fi\fi\fi\macros@XLIII@out}\newcommand\macros@XLIV[1][all]{\ifnum\pdfstrcmp{#1}{all}=0\def\macros@XLIV@out{\{"median": {-}0.97, "error plus": 0.044, "error minus": 0.016, "5th percentile": {-}0.99, "95th percentile": {-}0.93, "10th percentile": {-}0.98, "90th percentile": {-}0.94\}}\else\ifnum\pdfstrcmp{#1}{median}=0\def\macros@XLIV@out{{-}0.97}\else\ifnum\pdfstrcmp{#1}{error plus}=0\def\macros@XLIV@out{0.044}\else\ifnum\pdfstrcmp{#1}{error minus}=0\def\macros@XLIV@out{0.016}\else\ifnum\pdfstrcmp{#1}{5th percentile}=0\def\macros@XLIV@out{{-}0.99}\else\ifnum\pdfstrcmp{#1}{95th percentile}=0\def\macros@XLIV@out{{-}0.93}\else\ifnum\pdfstrcmp{#1}{10th percentile}=0\def\macros@XLIV@out{{-}0.98}\else\ifnum\pdfstrcmp{#1}{90th percentile}=0\def\macros@XLIV@out{{-}0.94}\else\def\macros@XLIV@out{??}\fi\fi\fi\fi\fi\fi\fi\fi\macros@XLIV@out}\newcommand\macros@XLV[1][all]{\ifnum\pdfstrcmp{#1}{all}=0\def\macros@XLV@out{\{"median": 0.98, "error plus": 0.011, "error minus": 0.027, "5th percentile": 0.95, "95th percentile": 0.99, "10th percentile": 0.96, "90th percentile": 0.99\}}\else\ifnum\pdfstrcmp{#1}{median}=0\def\macros@XLV@out{0.98}\else\ifnum\pdfstrcmp{#1}{error plus}=0\def\macros@XLV@out{0.011}\else\ifnum\pdfstrcmp{#1}{error minus}=0\def\macros@XLV@out{0.027}\else\ifnum\pdfstrcmp{#1}{5th percentile}=0\def\macros@XLV@out{0.95}\else\ifnum\pdfstrcmp{#1}{95th percentile}=0\def\macros@XLV@out{0.99}\else\ifnum\pdfstrcmp{#1}{10th percentile}=0\def\macros@XLV@out{0.96}\else\ifnum\pdfstrcmp{#1}{90th percentile}=0\def\macros@XLV@out{0.99}\else\def\macros@XLV@out{??}\fi\fi\fi\fi\fi\fi\fi\fi\macros@XLV@out}\newcommand\macros@XLVI[1][all]{\ifnum\pdfstrcmp{#1}{all}=0\def\macros@XLVI@out{\{"median": 0.2, "error plus": 0.8, "error minus": 0.77, "5th percentile": {-}0.58, "95th percentile": 1.0, "10th percentile": {-}0.34, "90th percentile": 0.84\}}\else\ifnum\pdfstrcmp{#1}{median}=0\def\macros@XLVI@out{0.2}\else\ifnum\pdfstrcmp{#1}{error plus}=0\def\macros@XLVI@out{0.8}\else\ifnum\pdfstrcmp{#1}{error minus}=0\def\macros@XLVI@out{0.77}\else\ifnum\pdfstrcmp{#1}{5th percentile}=0\def\macros@XLVI@out{{-}0.58}\else\ifnum\pdfstrcmp{#1}{95th percentile}=0\def\macros@XLVI@out{1.0}\else\ifnum\pdfstrcmp{#1}{10th percentile}=0\def\macros@XLVI@out{{-}0.34}\else\ifnum\pdfstrcmp{#1}{90th percentile}=0\def\macros@XLVI@out{0.84}\else\def\macros@XLVI@out{??}\fi\fi\fi\fi\fi\fi\fi\fi\macros@XLVI@out}\newcommand\macros@XLVII[1][all]{\ifnum\pdfstrcmp{#1}{all}=0\def\macros@XLVII@out{\{"median": 0.15, "error plus": 0.46, "error minus": 0.45, "5th percentile": {-}0.3, "95th percentile": 0.61, "10th percentile": {-}0.2, "90th percentile": 0.51\}}\else\ifnum\pdfstrcmp{#1}{median}=0\def\macros@XLVII@out{0.15}\else\ifnum\pdfstrcmp{#1}{error plus}=0\def\macros@XLVII@out{0.46}\else\ifnum\pdfstrcmp{#1}{error minus}=0\def\macros@XLVII@out{0.45}\else\ifnum\pdfstrcmp{#1}{5th percentile}=0\def\macros@XLVII@out{{-}0.3}\else\ifnum\pdfstrcmp{#1}{95th percentile}=0\def\macros@XLVII@out{0.61}\else\ifnum\pdfstrcmp{#1}{10th percentile}=0\def\macros@XLVII@out{{-}0.2}\else\ifnum\pdfstrcmp{#1}{90th percentile}=0\def\macros@XLVII@out{0.51}\else\def\macros@XLVII@out{??}\fi\fi\fi\fi\fi\fi\fi\fi\macros@XLVII@out}\newcommand\macros@XLVIII[1][all]{\ifnum\pdfstrcmp{#1}{all}=0\def\macros@XLVIII@out{\{"median": 0.44, "error plus": 0.12, "error minus": 0.13, "5th percentile": 0.31, "95th percentile": 0.57, "10th percentile": 0.34, "90th percentile": 0.54\}}\else\ifnum\pdfstrcmp{#1}{median}=0\def\macros@XLVIII@out{0.44}\else\ifnum\pdfstrcmp{#1}{error plus}=0\def\macros@XLVIII@out{0.12}\else\ifnum\pdfstrcmp{#1}{error minus}=0\def\macros@XLVIII@out{0.13}\else\ifnum\pdfstrcmp{#1}{5th percentile}=0\def\macros@XLVIII@out{0.31}\else\ifnum\pdfstrcmp{#1}{95th percentile}=0\def\macros@XLVIII@out{0.57}\else\ifnum\pdfstrcmp{#1}{10th percentile}=0\def\macros@XLVIII@out{0.34}\else\ifnum\pdfstrcmp{#1}{90th percentile}=0\def\macros@XLVIII@out{0.54}\else\def\macros@XLVIII@out{??}\fi\fi\fi\fi\fi\fi\fi\fi\macros@XLVIII@out}\newcommand\macros@XLIX[1][all]{\ifnum\pdfstrcmp{#1}{all}=0\def\macros@XLIX@out{\{"median": {-}0.98, "error plus": 0.036, "error minus": 0.013, "5th percentile": {-}0.99, "95th percentile": {-}0.94, "10th percentile": {-}0.99, "90th percentile": {-}0.95\}}\else\ifnum\pdfstrcmp{#1}{median}=0\def\macros@XLIX@out{{-}0.98}\else\ifnum\pdfstrcmp{#1}{error plus}=0\def\macros@XLIX@out{0.036}\else\ifnum\pdfstrcmp{#1}{error minus}=0\def\macros@XLIX@out{0.013}\else\ifnum\pdfstrcmp{#1}{5th percentile}=0\def\macros@XLIX@out{{-}0.99}\else\ifnum\pdfstrcmp{#1}{95th percentile}=0\def\macros@XLIX@out{{-}0.94}\else\ifnum\pdfstrcmp{#1}{10th percentile}=0\def\macros@XLIX@out{{-}0.99}\else\ifnum\pdfstrcmp{#1}{90th percentile}=0\def\macros@XLIX@out{{-}0.95}\else\def\macros@XLIX@out{??}\fi\fi\fi\fi\fi\fi\fi\fi\macros@XLIX@out}\newcommand\macros@L[1][all]{\ifnum\pdfstrcmp{#1}{all}=0\def\macros@L@out{\{"median": 0.98, "error plus": 0.013, "error minus": 0.031, "5th percentile": 0.94, "95th percentile": 0.99, "10th percentile": 0.95, "90th percentile": 0.99\}}\else\ifnum\pdfstrcmp{#1}{median}=0\def\macros@L@out{0.98}\else\ifnum\pdfstrcmp{#1}{error plus}=0\def\macros@L@out{0.013}\else\ifnum\pdfstrcmp{#1}{error minus}=0\def\macros@L@out{0.031}\else\ifnum\pdfstrcmp{#1}{5th percentile}=0\def\macros@L@out{0.94}\else\ifnum\pdfstrcmp{#1}{95th percentile}=0\def\macros@L@out{0.99}\else\ifnum\pdfstrcmp{#1}{10th percentile}=0\def\macros@L@out{0.95}\else\ifnum\pdfstrcmp{#1}{90th percentile}=0\def\macros@L@out{0.99}\else\def\macros@L@out{??}\fi\fi\fi\fi\fi\fi\fi\fi\macros@L@out}\newcommand\macros@LI[1][all]{\ifnum\pdfstrcmp{#1}{all}=0\def\macros@LI@out{\{"median": {-}0.081, "error plus": 1.1, "error minus": 0.92, "5th percentile": {-}1.0, "95th percentile": 1.0, "10th percentile": {-}0.94, "90th percentile": 0.83\}}\else\ifnum\pdfstrcmp{#1}{median}=0\def\macros@LI@out{{-}0.081}\else\ifnum\pdfstrcmp{#1}{error plus}=0\def\macros@LI@out{1.1}\else\ifnum\pdfstrcmp{#1}{error minus}=0\def\macros@LI@out{0.92}\else\ifnum\pdfstrcmp{#1}{5th percentile}=0\def\macros@LI@out{{-}1.0}\else\ifnum\pdfstrcmp{#1}{95th percentile}=0\def\macros@LI@out{1.0}\else\ifnum\pdfstrcmp{#1}{10th percentile}=0\def\macros@LI@out{{-}0.94}\else\ifnum\pdfstrcmp{#1}{90th percentile}=0\def\macros@LI@out{0.83}\else\def\macros@LI@out{??}\fi\fi\fi\fi\fi\fi\fi\fi\macros@LI@out}\newcommand\macros@LII[1][all]{\ifnum\pdfstrcmp{#1}{all}=0\def\macros@LII@out{\{"median": {-}0.025, "error plus": 0.54, "error minus": 0.52, "5th percentile": {-}0.54, "95th percentile": 0.52, "10th percentile": {-}0.43, "90th percentile": 0.4\}}\else\ifnum\pdfstrcmp{#1}{median}=0\def\macros@LII@out{{-}0.025}\else\ifnum\pdfstrcmp{#1}{error plus}=0\def\macros@LII@out{0.54}\else\ifnum\pdfstrcmp{#1}{error minus}=0\def\macros@LII@out{0.52}\else\ifnum\pdfstrcmp{#1}{5th percentile}=0\def\macros@LII@out{{-}0.54}\else\ifnum\pdfstrcmp{#1}{95th percentile}=0\def\macros@LII@out{0.52}\else\ifnum\pdfstrcmp{#1}{10th percentile}=0\def\macros@LII@out{{-}0.43}\else\ifnum\pdfstrcmp{#1}{90th percentile}=0\def\macros@LII@out{0.4}\else\def\macros@LII@out{??}\fi\fi\fi\fi\fi\fi\fi\fi\macros@LII@out}\newcommand\macros@LIII[1][all]{\ifnum\pdfstrcmp{#1}{all}=0\def\macros@LIII@out{\{"median": 0.5, "error plus": 0.19, "error minus": 0.19, "5th percentile": 0.31, "95th percentile": 0.69, "10th percentile": 0.35, "90th percentile": 0.65\}}\else\ifnum\pdfstrcmp{#1}{median}=0\def\macros@LIII@out{0.5}\else\ifnum\pdfstrcmp{#1}{error plus}=0\def\macros@LIII@out{0.19}\else\ifnum\pdfstrcmp{#1}{error minus}=0\def\macros@LIII@out{0.19}\else\ifnum\pdfstrcmp{#1}{5th percentile}=0\def\macros@LIII@out{0.31}\else\ifnum\pdfstrcmp{#1}{95th percentile}=0\def\macros@LIII@out{0.69}\else\ifnum\pdfstrcmp{#1}{10th percentile}=0\def\macros@LIII@out{0.35}\else\ifnum\pdfstrcmp{#1}{90th percentile}=0\def\macros@LIII@out{0.65}\else\def\macros@LIII@out{??}\fi\fi\fi\fi\fi\fi\fi\fi\macros@LIII@out}\newcommand\macros@LIV[1][all]{\ifnum\pdfstrcmp{#1}{all}=0\def\macros@LIV@out{\{"median": 0.77, "error plus": 0.11, "error minus": 0.17\}}\else\ifnum\pdfstrcmp{#1}{median}=0\def\macros@LIV@out{0.77}\else\ifnum\pdfstrcmp{#1}{error plus}=0\def\macros@LIV@out{0.11}\else\ifnum\pdfstrcmp{#1}{error minus}=0\def\macros@LIV@out{0.17}\else\def\macros@LIV@out{??}\fi\fi\fi\fi\macros@LIV@out}\newcommand\macros@LV[1][all]{\ifnum\pdfstrcmp{#1}{all}=0\def\macros@LV@out{\{"median": 77, "error plus": 11, "error minus": 17\}}\else\ifnum\pdfstrcmp{#1}{median}=0\def\macros@LV@out{77}\else\ifnum\pdfstrcmp{#1}{error plus}=0\def\macros@LV@out{11}\else\ifnum\pdfstrcmp{#1}{error minus}=0\def\macros@LV@out{17}\else\def\macros@LV@out{??}\fi\fi\fi\fi\macros@LV@out}\newcommand\macros@LVI[1][all]{\ifnum\pdfstrcmp{#1}{all}=0\def\macros@LVI@out{\{"median": 0.21, "error plus": 0.16, "error minus": 0.1\}}\else\ifnum\pdfstrcmp{#1}{median}=0\def\macros@LVI@out{0.21}\else\ifnum\pdfstrcmp{#1}{error plus}=0\def\macros@LVI@out{0.16}\else\ifnum\pdfstrcmp{#1}{error minus}=0\def\macros@LVI@out{0.1}\else\def\macros@LVI@out{??}\fi\fi\fi\fi\macros@LVI@out}\newcommand\macros@LVII[1][all]{\ifnum\pdfstrcmp{#1}{all}=0\def\macros@LVII@out{\{"median": 21, "error plus": 16, "error minus": 10\}}\else\ifnum\pdfstrcmp{#1}{median}=0\def\macros@LVII@out{21}\else\ifnum\pdfstrcmp{#1}{error plus}=0\def\macros@LVII@out{16}\else\ifnum\pdfstrcmp{#1}{error minus}=0\def\macros@LVII@out{10}\else\def\macros@LVII@out{??}\fi\fi\fi\fi\macros@LVII@out}\newcommand\macros@LVIII[1][all]{\ifnum\pdfstrcmp{#1}{all}=0\def\macros@LVIII@out{\{"median": 0.011, "error plus": 0.058, "error minus": 0.0096\}}\else\ifnum\pdfstrcmp{#1}{median}=0\def\macros@LVIII@out{0.011}\else\ifnum\pdfstrcmp{#1}{error plus}=0\def\macros@LVIII@out{0.058}\else\ifnum\pdfstrcmp{#1}{error minus}=0\def\macros@LVIII@out{0.0096}\else\def\macros@LVIII@out{??}\fi\fi\fi\fi\macros@LVIII@out}\newcommand\macros@LIX[1][all]{\ifnum\pdfstrcmp{#1}{all}=0\def\macros@LIX@out{\{"median": 1.1, "error plus": 5.8, "error minus": 0.96\}}\else\ifnum\pdfstrcmp{#1}{median}=0\def\macros@LIX@out{1.1}\else\ifnum\pdfstrcmp{#1}{error plus}=0\def\macros@LIX@out{5.8}\else\ifnum\pdfstrcmp{#1}{error minus}=0\def\macros@LIX@out{0.96}\else\def\macros@LIX@out{??}\fi\fi\fi\fi\macros@LIX@out}\makeatother
\usepackage{multirow}
\newcommand{\CIPlusMinus}[1]{{#1[median]^{+#1[error plus]}_{-#1[error minus]}}}
\newcommand{\CIPlusMinusPer}[1]{{#1[median]\%^{+#1[error plus]\%}_{-#1[error minus]\%}}}
\newcommand{\CIBoundsBracket}[1]{{[#1[5th percentile], #1[95th percentile]]}}
\newcommand{\CIBoundsDash}[1]{{#1[5th percentile]\textendash#1[95th percentile]}}

\newcommand{\msun}{\ensuremath{{\,\rm M}_\odot}}
\newcommand{\popA}{\textsc{SpinPop\textsubscript{A}}}
\newcommand{\popB}{\textsc{SpinPop\textsubscript{B}}}
\newcommand{\first}{\popA{}\textsc{:Peak}}
\newcommand{\contB}{\popB{}\textsc{:Continuum}}
\newcommand{\contA}{\popA{}\textsc{:Continuum}}
\newcommand{\brucepaper}{\citet{2022arXiv221012834E}}
\newcommand{\othreea}{\citet{2021ApJ...913L...7A}}
\newcommand{\cont}[1]{\textcolor{red}{DELTE THIS}}
\newcommand{\second}[1]{\textcolor{red}{DELTE THIS}}
\newcommand{\base}{\textsc{Isolated Peak Model}}
\newcommand{\comp}{\textsc{Peak+Continuum Model}}


\newcommand{\result}[1]{\textcolor{BurntOrange}{#1}}
\newcommand{\bruce}[1]{\textcolor{blue}{BE: #1}}
\newcommand{\jaxen}[1]{\textcolor{Cerulean}{JG: #1}}
\newcommand{\ben}[1]{\textcolor{orange}{BF: #1}}
\newcommand{\NewChange}[1]{\textcolor{BrickRed}{#1}}
\newcommand{\remove}[1]{\textcolor{WildStrawberry}{REMOVE: #1}}




\graphicspath{{./}{figures/}}

\begin{document}
\title{Cosmic Cousins: Identification of a Subpopulation of Binary Black Holes Consistent with Isolated Binary Evolution}

\author{Jaxen Godfrey}
\email{jaxeng@uoregon.edu}
\affiliation{Institute  for  Fundamental  Science, Department of Physics, University of Oregon, Eugene, OR 97403, USA}
\author{Bruce Edelman}
\affiliation{Institute  for  Fundamental  Science, Department of Physics, University of Oregon, Eugene, OR 97403, USA}
\author{Ben Farr}
\affiliation{Institute  for  Fundamental  Science, Department of Physics, University of Oregon, Eugene, OR 97403, USA}

% Abstract with filler text
\begin{abstract}
    Observations of gravitational waves (GWs) from merging compact binaries have become a regular occurrence. The continued advancement of the LIGO-Virgo-KAGRA (LVK) detectors have now produced a catalog of over 90 such mergers, from which we can begin to uncover the formation history of merging compact binaries. In this work, we search for subpopulations in the LVK's third gravitational wave transient catalog through the use of data-driven mixture models in a hierarchal Bayesian inference framework. By allowing for unique correlations between mass and spin in each subpopulation, we find an over density of mergers with a primary mass of $\result{\CIPlusMinus{\macros[Mass][Base][PeakA][max]}\msun}$, consistent with isolated binary formation \NewChange{and stable mass transfer}. This low-mass subpopulation has a spin magnitude distribution peaking at $a_\mathrm{peak}=$ \result{$\CIPlusMinus{\macros[SpinMag][Base][PeakA][max]}$}, exhibits spins preferentially aligned with the binary's orbital angular momentum, is constrained by \result{$\CIPlusMinus{\macros[NumEvents][Base][PeakA]}$} of our observations, and contributes \result{$\CIPlusMinusPer{\macros[BranchingRatios][Base][PeakA][Percent]}$} to the overall population of BBHs. We find a $\sim\macros[FracCut][BgreaterA]\%$ chance that the bulk of the events in the $15-55\msun$ range share a spin distribution with the $10\msun$ peak, with $99\%$ of these events possessing primary masses less than $m_{1,99\%} = $ \result{$\CIPlusMinus{\macros[Mass][Composite][ContinuumA][99percentile]}\msun$}, providing an estimate of the lower edge of the theorized pair instability mass gap. Additionally, we find mild evidence for a subpopulation of high mass BBHs near $60\msun$. This work is a first step in gaining a deeper understanding of compact binary formation and evolution with data-driven models, and will provide more robust conclusions as the catalog of observations becomes larger. 
\end{abstract}

% Main body with filler text
\section{Introduction} \label{sec:intro}

The first detection of gravitational waves (GWs) from a binary black hole (BBH) merger was made by the LIGO-Virgo-KAGRA (LVK) Collaboration on September 14, 2015. Since that fateful day, the LVK has detected nearly 100 compact binary coalescences (CBCs), bringing the third gravitational wave transient catalog (GWTC-3) up to 90 such events. \citep{2015CQGra..32g4001L,2015CQGra..32b4001A,2021PTEP.2021eA102A,2016PhRvL.116f1102A,2019PhRvX...9c1040A,2021PhRvX..11b1053A,2021arXiv211103606T}. With the maturation of GW Astronomy, novel studies of the universe are possible; we are now able to probe the entire population of merging compact objects in the universe with much greater fidelity than with the sparse, early LVK catalogs \citep{2019ApJ...882L..24A,2021ApJ...913L...7A,2021arXiv211103634T}. By breaking down the full CBC population into subpopulations based on different source properties- and paired with our theoretical knowledge of stellar astrophysics- we can begin to uncover the formation and evolution of compact binaries \citep{2017ApJ...846...82Z}. The two most common expected formation channels of merging compact objects are isolated formation and dynamical assembly, each predicted to produce binary populations with unique mass and spin characteristics \citep{2017Natur.548..426F,2018ApJ...854L...9F,10.3847/1538-4357/ab88b2}. The uncertainty in modeled merger rates of each formation channel is large, and the predictions continue to evolve with better understanding of the underlying physics (see \citet{10.1007/s41114-021-00034-3} for a thorough review on both modeled and observed merger rates of compact objects). By looking deeper at the correlations between the source properties at a population level, we can begin to look for subpopulations of observations that we can confidently associate with a specific formation channel.

Isolated formation of compact binaries occurs in galactic fields, where two gravitationally bound stars isolated from their environment undergo standard main sequence evolution, each eventually forming into a compact object. Energy loss from GW radiation causes the binary to inspiral, which can eventually lead to merger; however, in order for the binary to merge within a Hubble time, the initial orbital separation of the compact objects must be small \citep{10.1051/0004-6361/201936204,10.1007/s41114-021-00034-3}. Some process during stellar evolution is therefore required to rapidly decrease the orbital separation down to this limit for systems with much larger initial separations. One proposed mechanism is the ``common envelope phase'', in which a cloud of non-rotating gas engulfs the two objects (typically after one star has already collapsed into a compact object) and drag forces quickly dissipate orbital energy, thus reducing the orbital separation enough so that the resulting compact binary can merge due to GW emission alone \citep{10.1038/nature18322}. In dynamical formation scenarios, scattering or exchange interactions between astrophysical bodies in a dense stellar environment are thought to produce binaries capable of merging within a Hubble time \citep{1602.02444}. There are many theorized models for the main physical processes that contribute to isolated and dynamical formation, but there are few robust and direct predictions of observable quantities from these models. Instead, current predictions of merger rates and population distributions are estimated from numerical simulations, which have large uncertainties due to uncertain underlying physics or poorly constrained initial conditions \citep{10.1007/s41114-021-00034-3, 10.1051/0004-6361/201936204, 1806.00001v3, 1308.1546}. 

\begin{figure}[ht!]
    \begin{centering}
        \includegraphics[width=\linewidth]{figures/reweighed_catalog_m1m2.pdf}
        \caption{}
        \label{fig:mass_kde}
    \end{centering}
    \script{kde_contour_plot.py}
\end{figure}

\begin{figure}[ht!]
    \begin{centering}
        \includegraphics[width=\linewidth]{figures/reweighed_catalog_a1_tilt1.pdf}
        \caption{}
        \label{fig:spin_kde}
    \end{centering}
    \script{spin_contour_plot.py}
\end{figure}


The spin distribution of merging binaries is thought to provide the most direct evidence of their formation channel \citep{2017Natur.548..426F,2018ApJ...854L...9F}. Isolated binary evolution scenarios predict component spins to be near zero and preferentially aligned with the orbital angular momentum of the binary, though there are processes, such as angular momentum transport and supernova kicks, that can impart a small non-zero and modestly misaligned spin to one or both of the binary objects \citep{2203.02515, 1706.07053, 10.1051/0004-6361/201936204, 10.1051/0004-6361/202039804}. On the other hand, systems assembled dynamically in stellar clusters are thought to have no preferential alignment, producing an isotropically distributed spin tilt distribution \citep{10.3847/2041-8205/832/1/L2,10.1103/PhysRevD.100.043027}. With current data it is difficult to distinguish between these two channels, though studies have at least shown that GWTC-3 is not consistent with entirely dynamical or entirely isolated formation \citep{2021arXiv211103634T,2022ApJ...937L..13C,2022arXiv220902206T,2022arXiv221012834E,10.3847/2041-8213/ac86c4,10.48550/arXiv.2011.09570}. Recent studies have found support for a significant contribution of systems formed through dynamical assembly in the population of BBHs inferred from the GWTC-2 and GWTC-3 catalogs \citep{2021ApJ...913L...7A,2021PhRvD.104h3010R,2021arXiv211103634T,2022ApJ...937L..13C,2021ApJ...921L..15G,2022arXiv220902206T,2022arXiv220906978V,2022arXiv221012834E}, though with large uncertainties. 

While spin may be the characteristic most directly linked to compact binary formation history, the LVK parameter estimation of individual event spin properties contains large uncertainties, making it difficult to disentangle competing formation channels with spin alone. However, the component masses of individual events are typically inferred with greater certainty than their spin, and there are even features in the mass distribution that may signal the existence of different subpopulations \citep{2021ApJ...913L..19T,2022ApJ...924..101E,2021arXiv211103634T,2022ApJ...928..155T,2022arXiv221012834E}. Unfortunately, it can also be challenging to distinguish between the isolated and dynamical formation channels using only component mass, as the models in both scenarios predict masses that significantly overlap \citep{1609.05916}. Instead, a search for correlated population properties across mass, spin, and redshift may prove to be much more fruitful in distinguishing between the different CBC formation channels \citep{2021ApJ...912...98F,2021ApJ...922L...5C,2022ApJ...931...17V,2022ApJ...932L..19B}. 

In this letter, we search for signs of possible BBH subpopulations in GWTC-3 by incorporating discrete latent variables in the hierarchical Bayesian inference framework to probabilistically assign each BBH observation into separate categories that are associated with distinctly different mass and spin distributions. Incorporating these discrete variables during inference allows us to easily infer each BBH's association with each category, in addition to the posterior distributions for astrophysical branching ratios. The remaining sections of letter are structured as follows: Section \ref{sec:methods} describes the statistical framework with the inclusion of discrete latent variables and the specific models used for separate subpopulations. Section \ref{sec:results} presents the results of our study, including the inferred branching ratios and the inferred subpopulation membership probabilities for each BBH in GWTC-3. In section \ref{sec:astro} we discuss the implications of our findings and how it relates to the current understanding of compact binary formation and population synthesis. We finish in section \ref{sec:conclusion}, with a summary of the letter and prospects for distinguishing subpopulations in future catalogs after the LVK's fourth observing run. 
\section{Methods} \label{sec:methods}

\subsection{Statistical Framework} \label{sec:statistical_framework}

We employ the typical hierarchical Bayesian inference framework to infer the properties of the population of merging compact binaries given a catalog of observations. The rate of compact binary mergers is modeled as an inhomogeneous Poisson point process \citep{10.1093/mnras/stz896}, with the merger rate per comoving volume $V_c$ \citep{astro-ph/9905116}, source-frame time $T_\text{src}$ and binary parameters $\theta$ defined as: 

\begin{equation} \label{eq:rate}
    \frac{dN}{dV_cdt_\mathrm{src}d\theta} = \frac{dN}{dV_cdt_\mathrm{src}} p(\theta | \Lambda) = \mathcal{R} p(\theta | \Lambda)
\end{equation}

\noindent with $p(\theta | \Lambda)$ the population model, $\mathcal{R}$ the merger rate, and $\Lambda$ the set of population hyperparameters. Following other population studies \citep{10.1093/mnras/stz896,2021ApJ...913L...7A,2111.03634,2007.05579}, we use the hierarchical likelihood \citep{10.1063/1.1835214} that incorporates selection effects and marginalizes over the merger rate as: 

\begin{equation} \label{eq:likelihood}
    \mathcal{L}(\bm{d} | \Lambda) \propto \frac{1}{\xi(\Lambda)} \prod_{i=1}^{N_\mathrm{det}} \int d\theta \mathcal{L}(d_i | \theta) p(\theta | \Lambda)
\end{equation}

\noindent Above, $\bm{d}$ is the set of data containing $N_\mathrm{det}$ observed events, $\mathcal{L}(d_i | \theta)$ is the individual event likelihood function for the $i$th event given parameters $\theta$ and $\xi(\Lambda)$ is the fraction of merging binaries we expect to detect, given a population described by $\Lambda$. The integral of the individual event likelihoods marginalizes over the uncertainty in each event's binary parameter estimation, and is calculated with Monte Carlo integration and by importance sampling, reweighing each set of posterior samples to the likelihood. The detection fraction is calculated with:

\begin{equation} \label{eq:detfrac}
    \xi(\Lambda) = \int d\theta p_\mathrm{det}(\theta) p(\theta | \Lambda)
\end{equation}

\noindent with $p_\mathrm{det}(\theta)$ the probability of detecting a binary merger with parameters $\theta$. We calculate this fraction using simulated compact merger signals that were evaluated with the same search algorithms that produced the catalog of observations. With the signals that were successfully detected, we again use Monte Carlo integration to get the overall detection efficiency, $\xi(\Lambda)$ \citep{1712.00482, 1904.10879, 2204.00461}.

To model multiple subpopulations, we incorporate discrete latent variables to probabilistically assign events to categories with unique mass and spin distributions. \jaxen{Addressing Tom's 4th comment: This is essentially an extension of the foreground/background signal categorization developed in \citet{1302.5341}, applied hierarchical inference.} Incorporating these discrete variables during inference allows us to easily infer each BBH's association with each category, in addition to the posterior distributions for astrophysical branching ratios (also referred to as mixing fractions). Previous applications of mixture models to the binary catalog \citep{10.3847/2041-8213/abe949,2022arXiv220902206T} implicitly marginalize over event categories. The population properties inferred using these approaches are identical, and event categorization probabilities can be calculated, but sampling these latent variables enables detailed investigation of correlations between categorization of events and population properties. 

For $M$ subpopulations in a catalog of $N_\mathrm{det}$ detections, we add a latent variable $k_i$ for each merger that can be $M$ different discrete values between $0$ and $M-1$, each associated with a separate model, $p_{k_i}(\theta | \Lambda_{k_i})$, and hyperparameters, $\Lambda_{k_i}$. Evaluating the model (or hyper-prior) for the $i^\mathrm{th}$ event with binary parameters, $\theta_i$, given latent variable $k_i$ and hyperparameters $\Lambda_{k_i}$, we have:

\begin{equation} \label{eq:latent}
    p(\theta_i | \Lambda, k_i) = p_{k_i}(\theta_i | \Lambda_{k_i})
\end{equation}

\noindent To construct our probabilistic model, we first sample $p_{\lambda} \sim \mathcal{D}(M)$, from an M-dimensional Dirichlet distribution of equal weights, representing the astrophysical branching ratios $\lambda_{k_i}$ of each subpopulation. Then each of the $N_\mathrm{det}$ discrete latent variables are sampled from a categorical distribution with each category, $k_i$, having probability $p_{\lambda_{k_i}}$. Within the \textsc{NumPyro} \citep{1810.09538,1912.11554} probabilistic programming language, we use the implementation of \texttt{DiscreteHMCGibbs} \citep{Liu1996PeskunsTA} to sample the discrete latent variables, while using the \texttt{NUTS} \citep{1111.4246} sampler for continuous variables. While this approach may seem computationally expensive, we find that the conditional distributions over discrete latent variables enable Gibbs sampling with similar costs and speeds to the equivalent approach that marginalizes over each discrete latent variable, $k_i$. We find the same results with either approach and only slight performance differences that depend on specific model specifications, and thus opt for the approach without marginalization. This method also has the advantage that we get posterior distributions on each event's subpopulation assignment without extra steps.

\subsection{Astrophysical Mixture Models} \label{sec:astromodels} 

Given the recent evidence for peaks in the BBH primary mass spectrum at $10 M_{\odot}$, $35 M_{\odot}$, and suggestions of a potential feature at $\sim20 M_{\odot}$ \citep{10.3847/2041-8213/aa9bf6, 10.3847/1538-4357/aab34c, 10.3847/2041-8213/ab3800, 2021ApJ...913L...7A, 2111.03634, 2022ApJ...928..155T,2022arXiv221012834E}, we chose models similar to the \textsc{Multi Spin} model in \cite{2021ApJ...913L...7A}, which is characterised by a power-law plus a Gaussian peak in primary mass, wherein the spin distributions of the two mass components are allowed to differ from each other. In our case, we replace the power law mass components and parametric spin descriptions with non-parametric Basis-Spline functions, in order to avoid model dependent biases on the resulting distributions.
We also choose to parametrize peaks in the primary mass spectrum in terms of log-mass, rather than linear-mass. 
In both of our model prescriptions, detailed below, the spin magnitude and tilt distributions of each binary component are assumed to be independently and identically distributed (IID), i.e. we fit the same model distribution to both binary spin components per category. To reduce the number of free parameters and thus computational cost, we fix the power law slope of the merger rate with redshift to $\lambda_z=2.7$ and do not fit the mininum and maximum BBH primary mass, instead truncating all the mass distributions below $m_\text{min} = 5.0\msun$ and above $m_\text{max} = 100\msun$. We make use of the mass and spin basis spline (B-Spline) models from \cite{2022arXiv221012834E}. We fix the number of knots $n$ in all B-Spline models used, choosing $n_{m_1}=48$, $n_{q}=30$, $n_a=16$, and $n_{cos(\theta)}=16$. All the models and formalism used in our analysis are available in the \href{https://git.ligo.org/bruce.edelman/gwinferno}{GWInferno} python library, along with the code and data to reproduce this study in this GitHub \href{https://github.com/jaxeng/paper}{repository}. 



\subsubsection{\base{}}
In this model prescription, we categorize the BBH population into $M=2$ subpopulations, which we will refer to as \first{} and \contB{}, based on their primary mass distributions (continuum here referring to the non-parametric nature of the B-Spline distributions). The \first{} subpopulation is characterised by a truncated Log-Normal peak in primary mass and B-Spline functions in mass ratio, spin magnitude, and tilt angle. The \contB{} subpopulation is characterized by B-Spline functions in all mass and spin parameters. We infer the mean $\mu_m$ and standard deviation $\sigma_m$ of the LogNormal peak, along with the B-Spline coefficients $\mathbf{c}$ for all other parameters. During inference \first{} is associated with $k_i = 0$ and \contB{} is associated with $k_i = 1$.

\begin{itemize}
    \item \first{} (64 parameters), $k_i=0$. This category assumes a truncated Log-Normal model in primary mass and B-spline ($B_k$) models in mass ratio $q$, spin magnitude $a_j$, and $cos(\theta_j)$. Note that since we assume the spins to be IID, the B-Spline spin parameters, such as $\mathbf{c}_{a,0}$, are the same for each binary component $j=1$ and $j=2$
    \begin{eqnarray} \label{eq:peakAbase}
        p_{m,0}(m_1| \Lambda_{m,0}) = \text{Lognormal}_\text{T}(m_1 | \mu_{m}, \sigma_{m}) \\
        \text{log} p_{q,0}(q| \Lambda_{q,0}) = B_k(q | \mathbf{c}_{q,0}) \\
        \text{log} p_{a,0}(a_j| \Lambda_{a,0}) = B_k(a_j | \mathbf{c}_{a,0}) \\
        \text{log} p_{\theta,0}(cos(\theta_j)| \Lambda_{\theta,0}) = B_k( cos(\theta_j) | \mathbf{c}_{\theta,0})
    \end{eqnarray}

    \item \contB{} (110 parameters), $k_i=1$. The mass ratio and spin models have the same form as \first{}, but here primary mass is fit to a B-spline function. 
    \begin{eqnarray} \label{eq:contBbase}
        \text{log} p_{m,1}(m_1| \Lambda_{m,1}) = B_k(m_1 | \mathbf{c}_{m}) \\
        \text{log} p_{q,1}(q| \Lambda_{q,0}) = B_k(q | \mathbf{c}_{q,1}) \\
        \text{log} p_{a,1}(a_j| \Lambda_{a,1}) = B_k(a_j | \mathbf{c}_{a,1}) \\
        \text{log} p_{\theta,1}(cos(\theta_j)| \Lambda_{\theta,1}) = B_k( cos(\theta_j) | \mathbf{c}_{\theta,1})
    \end{eqnarray}

\end{itemize}

\subsubsection{\comp{}}

To investigate whether other parts of the primary mass spectrum may have similar spin characteristics to \first{}, we construct a composite subpopulation model, wherein we include an additional mass component \contA{}, described by a B-Spline in primary mass but force it to have the same spin distributions as \first{}. \contB{} is still included and is described by it's own spin distributions. For this model prescription, we chose to marginalize over the discrete variables $k_i$ during sampling, in order to sample slightly more efficiently. 

\begin{itemize}
    \item \first{} (64 parameters), $k_i=0$. This category assumes a truncated Log-Normal model in primary mass, and B-spline models in mass ratio $q$, spin magnitude $a_j$, and $cos(\theta_{j})$. 
    \begin{eqnarray} \label{eq:peakAcomp}
        p_{m,0}(m_1| \Lambda_{m,0}) = \text{Lognormal}_\text{T}(m_1 | \mu_{m}, \sigma_{m}) \\
        \text{log} p_{q,0}(q| \Lambda_{q,0}) = B_k(q | \mathbf{c}_{q,0}) \\
        \text{log} p_{a,0}(a_j| \Lambda_{a,0}) = B_k(a_j | \mathbf{c}_{a,0}) \\
        \text{log} p_{\theta,0}(cos(\theta_j)| \Lambda_{\theta,0}) = B_k( cos(\theta_j) | \mathbf{c}_{\theta,0})
    \end{eqnarray}

    \item \contA{} (78 parameters), $k_i=1$ for mass, $k_i = 0$ for spin. The mass ratio and spin models are the same as the previous category, but here primary mass is fit to a B-spline function. 
    \begin{eqnarray} \label{eq:contAcomp}
        \text{log} p_{m,1}(m_1| \Lambda_{m,1}) = B_k(m_1 | \mathbf{c}_{m, 1}) \\
        \text{log} p_{q,1}(q| \Lambda_{q,1}) = B_k(q | \mathbf{c}_{q,1}) \\
        \text{log} p_{a,1}(a_j| \Lambda_{a,1}) = B_k(a_j | \mathbf{c}_{a,0}) \\
        \text{log} p_{\theta,1}(cos(\theta_j)| \Lambda_{\theta,1}) = B_k( cos(\theta_j) | \mathbf{c}_{\theta,0})
    \end{eqnarray}

    \item \contB{} (110 parameters), $k_i=2$. Here, all mass and spin properties are fit to B-Splines.
    \begin{eqnarray} \label{eq:contBcomp}
        \text{log} p_{m,2}(m_1| \Lambda_{m,2}) = B_k(m_1 | \mathbf{c}_{m, 2}) \\
        \text{log} p_{q,2}(q| \Lambda_{q,2}) = B_k(q | \mathbf{c}_{q,2}) \\
        \text{log} p_{a,2}(a_j| \Lambda_{a,2}) = B_k(a_j | \mathbf{c}_{a,2}) \\
        \text{log} p_{\theta,2}(cos(\theta_j)| \Lambda_{\theta,2}) = B_k( cos(\theta_j) | \mathbf{c}_{\theta,2})
    \end{eqnarray}
\end{itemize}
\section{Results} \label{sec:results}

\begin{itemize}
    \item Start by introducing the dataset (GWTC-3) and threshold/cuts on catalog for our dataset
    \item Show results of main run model -- mass dist -- spin dists etc
    \item Discuss more specific details on different subpopulation mass/spin dists
    \item Talk about astrophysical branching ratios of subpopulations and which observations were "put" within each of the subpops
    \item Quantitative statements on spin mag dist of our isolated subpopulation
    \item Quantitative statements on spin orientation dist of our isolated subpop. How much does it prefer aligned spins over the other subpops?
\end{itemize}

\begin{figure*}[ht!]
    \begin{centering}
        \includegraphics[width=\linewidth]{figures/mass_distribution_plot.pdf}
        \caption{The marginal primary mass distribution}
        \label{fig:mass_distribution}
    \end{centering}
    \script{mass_distribution_plot.py}
\end{figure*}

Figure \ref{fig:mass_distribution} shows the inferred primary mass distribution. 
\section{Astrophysical Interpretation} \label{sec:astro}

\begin{itemize}
    \item What can this new identified subpop help to enlighten in stellar pop synth community?
    \item can we use spin tilt dist to make statements on supernovae kicks in isolated formation?
    \item How does this compare to LVK work and other recent work? Are our results consistent or in conflict with dyn/iso fractions?
    \item report fdyn / fhm and etc
\end{itemize}

\section{Conclusion} \label{sec:conclusion}

\begin{itemize}
    \item Reiterate the motivation of the work
    \item restate the main conclusions leading us to identify this 10 solar mass peak as isolated
    \item briefly comment on main astro implications from prev section
    \item Discuss further work on other ways we can use this method to probe formation channels even deeper. (use spin vs mass dist to disentangle the sub pops. i.e. isotropic tilt for dynamical -- aligned tilt for isolated)
    \item Discuss other applications of discrete latent variables (label for BNS/NSBH/BBH, label for 1G/2G/3G etc)
\end{itemize}

\section{Acknowledgements}\label{sec:acknowledments}
This material is based upon work supported by the National Science Foundation Graduate Research Fellowship under Grant No. 2236419. This research has made use of data, software and/or web tools obtained from the Gravitational Wave Open Science Center 
(\url{https://www.gw-openscience.org/}), a service of LIGO Laboratory, the LIGO Scientific Collaboration and the Virgo Collaboration. 
The authors are grateful for computational resources provided by the LIGO Laboratory and supported by National Science Foundation Grants PHY-0757058 and PHY-0823459.  
This work benefited from access to the University of Oregon high performance computer, Talapas. This material is based upon work supported 
in part by the National Science Foundation under Grant PHY-2146528 and work supported by NSF's LIGO Laboratory which is a major facility 
fully funded by the National Science Foundation.
\software{
\textsc{Showyourwork}~\citep{2110.06271},
\textsc{Astropy}~\citep{1307.6212, 1801.02634, 2206.14220},
\textsc{NumPy}~\citep{10.1038/s41586-020-2649-2},
\textsc{SciPy}~\citep{10.1038/s41592-019-0686-2},
\textsc{Matplotlib}~\citep{10.1109/MCSE.2007.55},
\textsc{Jax}~\citep{github.com/google/jax},
\textsc{NumPyro}~\citep{1810.09538,1912.11554},
}
\bibliography{bib}{}
\bibliographystyle{aasjournal}

% \variable{output/event_cat_table.tex}

\appendix

\begin{table*}[b!]
    \centering
    \begin{tabular}{|l|l|l|l|}
    \hline
    \textbf{Model} & \textbf{Parameter} & \textbf{Description} & \textbf{Prior} \\ \hline \hline
    \multicolumn{4}{|c|}{\textbf{Primary Mass Model Parameters}} \\ \hline


    \first & $\mu_m$ & Mean of Truncated Log Peak & $ \sim \mathrm{N}_{LT}(\ln10,\ln5, \text{a}=5) $ \\ \cline{2-4} 
    & $\sigma_m$ & Standard Deviation of Truncated Log Peak & $\sim \mathrm{N}_T(0, 0.2, \text{a}=0.01, \text{b}=0.3)$ \\ \hline

    \contA & $\bm{c}$ & Basis coefficients & $\sim \mathrm{Smooth}(\tau_\lambda, \sigma, r, n)$ \\ \cline{2-4} 
     and & $\tau_\lambda$ & Smoothing Prior Scale & 1 \\ \cline{2-4}
     \contB & $r$ & order of the difference matrix for the smoothing prior & 1 \\ \cline{2-4} 
     & $\sigma$ & width of Gaussian priors on coefficients in smoothing prior & 15 \\ \cline{2-4} 
     & $n$ & number of basis functions in the basis spline & 48 \\ \hline \hline 

     


    \multicolumn{4}{|c|}{\textbf{Mass Ratio Model Parameters}} \\ \hline
    All Mass Ratio & $\bm{c}$ & Basis coefficients & $\sim \mathrm{Smooth}(\tau_\lambda, \sigma, r, n)$ \\ \cline{2-4} 
    Models & $\tau_\lambda$ & Smoothing Prior Scale & 1 \\ \cline{2-4}
    & $r$ & order of the difference matrix for the smoothing prior & 1 \\ \cline{2-4} 
    & $\sigma$ & width of Gaussian priors on coefficients in smoothing prior & 5 \\ \cline{2-4} 
    & $n$ & number of basis functions in the basis spline & 30 \\ \hline \hline 



    \multicolumn{4}{|c|}{\textbf{Redshift Evolution Model Parameters}} \\ \hline
    All Redshift Models & $\lambda$ & slope of redshift evolution power law $(1+z)^\lambda$ &  $\sim \mathcal{N}(0,3)$ \\ \cline{2-4}
    & $\bm{c}$ & Basis coefficients & $\sim \mathrm{Smooth}(\tau_\lambda, \sigma, r, n)$ \\ \cline{2-4} 
     & $\tau_\lambda$ & Smoothing Prior Scale & 1 \\ \cline{2-4}
     & $r$ & order of the difference matrix for the smoothing prior & 2 \\ \cline{2-4} 
     & $\sigma$ & width of Gaussian priors on coefficients in smoothing prior & 10 \\ \cline{2-4} 
     & $n$ & number of knots in the basis spline & 20 \\ \hline \hline 



    \multicolumn{4}{|c|}{\textbf{Spin Distribution Model Parameters}} \\ \hline
    All Spin Magnitude & $\bm{c}$ &  Basis coefficients & $\sim \mathrm{Smooth}(\tau_\lambda, \sigma, r, n)$  \\ \cline{2-4} 
    Models & $\tau_\lambda$ & Smoothing Prior Scale & 25\\ \cline{2-4}
    & $r$ & order of the difference matrix for the smoothing prior & 2 \\ \cline{2-4} 
    & $\sigma$ & width of Gaussian priors on coefficients in smoothing prior & 5 \\ \cline{2-4} 
    & $n$ & number of knots in the basis spline & 16 \\ \hline \hline 


    All Tilt Models & $\bm{c}$ &  Basis coefficients & $\sim \mathrm{Smooth}(\tau_\lambda, \sigma, r, n)$  \\ \cline{2-4} 
    & $\tau_\lambda$ & Smoothing Prior Scale & 25 \\ \cline{2-4}
    & $r$ & order of the difference matrix for the smoothing prior & 2 \\ \cline{2-4} 
    & $\sigma$ & width of Gaussian priors on coefficients in smoothing prior & 5 \\ \cline{2-4} 
    & $n$ & number of knots in the basis spline & 16 \\ \hline \hline
    \end{tabular}
    \caption{All hyperparameter prior choices for each of the newly introduced basis spline models from this manuscript. See \brucepaper for more detailed description of basis spline or smoothing prior parameters.}
    \label{tab:model_priors}
\end{table*} 

\end{document}