\section{Astrophysical Interpretation} \label{sec:astro}

\begin{itemize}
    \item What can this new identified subpop help to enlighten in stellar pop synth community?
    \item can we use spin tilt dist to make statements on supernovae kicks in isolated formation?
    \item How does this compare to LVK work and other recent work? Are our results consistent or in conflict with dyn/iso fractions?
    \item report fdyn / fhm and etc
\end{itemize}

\section{Conclusion} \label{sec:conclusion}

\begin{itemize}
    \item Reiterate the motivation of the work
    \item restate the main conclusions leading us to identify this 10 solar mass peak as isolated
    \item briefly comment on main astro implications from prev section
    \item Discuss further work on other ways we can use this method to probe formation channels even deeper. (use spin vs mass dist to disentangle the sub pops. i.e. isotropic tilt for dynamical -- aligned tilt for isolated)
    \item Discuss other applications of discrete latent variables (label for BNS/NSBH/BBH, label for 1G/2G/3G etc)
\end{itemize}