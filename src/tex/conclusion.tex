\section{Astrophysical Interpretation} \label{sec:astro}

% \begin{itemize}
%     \item What can this new identified subpop help to enlighten in stellar pop synth community?
%     \item can we use spin tilt dist to make statements on supernovae kicks in isolated formation?
%     \item How does this compare to LVK work and other recent work? Are our results consistent or in conflict with dyn/iso fractions?
%     \item report fdyn / fhm and etc
% \end{itemize}

Spin predictions of binaries formed in the stellar field are dependent on the many physical processes that may occur prior to the stellar binary becoming a BBH.  Spin magnitude of a BH can depend sensitively on the efficiency of angular momentum (AM) transport between its progenitor stellar core and envelope \citep{2203.02515}. Efficient AM transport, such as through the Taylor-Spruit magnetic dynamo \citep{10.1051/0004-6361:20011465}, leads to low spinning BHs \citep{10.3847/2041-8213/ab339b} while less efficient AM transport, such as that predicted by the shellular model \citep{1992A&A...265..115Z,2012A&A...537A.146E,10.3847/1538-4365/aacb24,2019MNRAS.485.4641C}, can preserve the spin of the progenitor star. However, the effects of AM transport by these mechanisms can be obfuscated by tidal interactions between the binary components, accretion, and mass transfer, which can spin up the binary or increase AM transport efficiency, thus shedding spin. Natal supernova kicks are thought to be the leading cause of spin orbit misalignments in field binaries, while tidal forces and mass transfer tend to align BH spins with the orbital AM. As discussed in \citet{2023PhRvX..13a1048A}, recent studies of BBH formation in globular clusters predict \emph{suppressed} merger rates for binaries with primary BH masses $\lesssim 10 \msun$~\citep{2009.01861,10.1103/PhysRevD.100.043027,1808.04514}.  Therefore, BBH formation in globular clusters is unlikely to be responsible for the BBHs in the $10\msun$ peak, and is most likely a subdominant formation mechanism in the catalog.

\NewChange{When a star in an isolated binary overflows its Roche-Lobe, mass will be transferred onto its companion. If the binary is below a critical mass ratio, the mass transfer will be stable and the binary will gradually inspiral. If the mass ratio is large enough to cause unstable mass transfer, a common envelope can form, which will also allow the binary to inspiral. This stability dependence on mass ratio will impact the resulting BBH mass ratio distribution of these systems, as seen in predictions from \citet{2110.01634}. Figure 9 in \citet{2110.01634} shows the mass ratio distributions predicted by the common envelope and stable mass transfer channels. Notably, we find that the mass ratio distribution of the $10\msun$ peak in both our models reflects the shape of the stable mass transfer mass ratio distribution in \citet{2110.01634}.}

While we find that the $10\msun$ subpopulation is consistent with the general characteristics associated with field formation, large uncertainties in both spin measurements and predictions from population synthesis models prevent us from placing informative constraints on the formation physics. If the $10\msun$ subpopulation is indeed a product of isolated binary evolution, then our inferred spin distributions hint at this channel producing binaries with low, modestly misaligned spins. This could indicate that AM transport is efficient in massive stars, as modeled by variations of the Taylor-Spruit magnetic dynamo \citep{1706.07053}, and that natal kicks are a common occurrence during field BBH formation. \NewChange{Additionally, we find compelling evidence that suggests the $10\msun$ peak is a product of isolated binaries that have undergone stable mass transfer rather than a common envelope phase.}

Our results are consistent with other analyses that find the data does not require a discontinuous non-spinning subpopulation \citep{2022ApJ...937L..13C,2205.12329, 2022arXiv220902206T, 2301.01312,10.48550/arXiv.2302.07289, 2210.12287}, though is in tension with other analyses that have claimed its existence \citep{2021ApJ...921L..15G,2021PhRvD.104h3010R}; however, we cannot completely rule out a non-spinning population. The $20\msun$ and $35\msun$ peaks may also be consistent with field formation, as we infer spin properties similar to those of the $10\msun$ peak; though, as we discussed in Section \ref{sec:results} there is a possibility that some events in this mass range are more consistent with field formation spin characteristics. The astrophysical branching ratios we infer in the \comp{} are consistent with the field and dynamical branching ratios inferred by \citet{2011.10057}, $[\beta_{\text{field}}, \beta_{\text{dynamical}}] = [0.86^{+0.11}_{-0.36}, 0.14^{0.36}_{-0.11}]$, under the assumption that \popA{} is consistent with field formation and \contB{} is more consistent with dynamical formation.

The sharp fall-off in primary mass of \contA{} in the \comp{} could give an estimate of the lower edge of the PISN mass gap. The 99th percentile of the primary mass distribution of \contA{} is $m_{1,99\%} = $ \result{$\CIPlusMinus{\macros[Mass][Composite][ContinuumA][99percentile]}$ \msun}, which is consistent with predictions that place the lower edge of the gap between $40-70\msun$ \citep{1901.00215,1910.12874v1,2103.07933v1,2104.07783v2}.


\section{Conclusion} \label{sec:conclusion}

% \begin{table*}[ht!]
%     \centering
%     \begin{tabular}{@{}cccc@{}}
%     \toprule
%     Model & CYB & $\base{}$ & $\comp{}$ \\ \midrule
%     CYB & 0 & $-\macros[LogBayesFactors][IP_to_CYB]$ & $-\macros[LogBayesFactors][PC_to_CYB]$ \\
%     $\base{}$ & \macros[LogBayesFactors][IP_to_CYB] & 0 & $-\macros[LogBayesFactors][PC_to_IP]$ \\
%     $\comp{}$ & \macros[LogBayesFactors][PC_to_CYB] & \macros[LogBayesFactors][PC_to_IP] & 0 \\
%     \bottomrule
%     \end{tabular}
%     \caption{$\log_{10}$ Bayes factors of the \base{}, \comp{}, and the Cover Your Basis (CYB) model (\brucepaper). The values follow the format $\log_{10} \text{BF}_\text{row,col}.$; e.g. the \comp{} row and CYB column shows the $\log_{10}$ Bayes factor of the \comp{} relative to the CYB model.}
%     \label{tab:BF}
%     \end{table*}

As the catalog of compact object mergers continues to grow, we are able to probe the physical properties of these systems with higher fidelity and uncover details in their distributions previously obscured by our lack of data. With these advancements comes the ability to piece together formation histories imprinted in the details of these distributions. Understanding the physical properties of compact binaries and their formation has implications for the broader astrophysics community such as providing constraints on stellar evolution theories and population synthesis simulations, the physics of globular clusters, the impact of stellar metallicity, neutron star equations of state, and much more.

By leveraging the hierarchical Bayesian inference toolkit, a mixture of parametric and data-driven models, and combining information across mass and spin, we were able to identify a peak in the BBH primary mass spectrum at $m_\text{1,peak} = $ \result{$\CIPlusMinus{\macros[Mass][Base][PeakA][max]}$ \msun} that corresponds to a subpopulation of BBH's with low spins and a preference for alignment, consistent with isolated binary formation \NewChange{and stable mass transfer}. We then extended our \base{} to search the rest of the mass spectrum for events with similar spin characteristics to the $10\msun$ subpopulation. We found that the $20\msun$ and $35\msun$ peaks are consistent with the $10\msun$ events, though there \textit{may} be an underlying subpopulation consistent with the highest mass events. We find evidence of a distinct population at high masses whose spin properties are largely uninformed by the current catalog.

Due to the large uncertainties that are currently present in both population synthesis models and the measurements of BBH properties, we are unable to place strong constraints on the physics behind isolated binary evolution or (P)PISN. However, if the $10\msun$, $20\msun$, and $35\msun$ subpopulations are truly products of these channels, they likely produce binaries with low spins, though whether these formation channels are dominated by non-spinning BHs remains unclear. Aligned spin systems are also not completely ruled out, but the subpopulations appear to possess modest misalignments and the apparent fall-off of the $35\msun$ peak may indicate a lower bound on the PISN mass gap of \result{$\CIPlusMinus{\macros[Mass][Composite][ContinuumA][99percentile]}$ \msun}.

The data driven models used in this analysis and developed in the python library \textsc{GWInferno} can inform us about the full compact binary catalog beyond identifying BBH subpopulations. Currently, the LVK categorizes mergers as binary black holes, binary neutron stars, or neutron star binary black holes \emph{a priori} based on mass thresholds and tidal deformation values. Instead of categorizing merger components \emph{a priori} and then fitting the mass and spin distributions of each category individually, discrete latent variables could be used to simultaneously classify merger components and infer their mass and spin distributions.

In future work, we will incorporate a method for classifying outliers to subpopulations as well as testing this framework on a simulated catalog in order to better understand how uninformed data affects our results. Repeating this analysis on the data from the next observing run -- which is expected to substantially increase the current catalog size -- should provide more insights with stronger constraints on subpopulation properties.
