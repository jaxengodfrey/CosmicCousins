\section{Results} \label{sec:results}

\begin{figure*}[ht!]
    \begin{centering}
        \includegraphics[width=\linewidth]{figures/mass_distribution_plot.pdf}
        \caption{The marginal primary mass distribution}
        \label{fig:mass_distribution}
    \end{centering}
    \script{mass_distribution_plot.py}
\end{figure*}


\begin{itemize}
    \item Start by introducing the dataset (GWTC-3) and threshold/cuts on catalog for our dataset
    \item Show results of main run model -- mass dist -- spin dists etc
    \item Discuss more specific details on different subpopulation mass/spin dists
    \item Talk about astrophysical branching ratios of subpopulations and which observations were "put" within each of the subpops
    \item Quantitative statements on spin mag dist of our isolated subpopulation
    \item Quantitative statements on spin orientation dist of our isolated subpop. How much does it prefer aligned spins over the other subpops?
\end{itemize}

With these models and framework in hand, we infer the mass and spin distributions with the recently released LVK catalog of gravitational wave observations, GWTC-3 \jaxen{CITE THIS}. We perform the same BBH threshold cuts on the catalog done by the LVK's accompnaying population analysis \jaxen{CITE THIS}, which leaves us with 70 BBH mergers. Additionally, we choose to remove GW190814 from our anlaysis, as it is likely an outlier of the total BBH population and is not very well understood \jaxen{CITE THIS}. With the 69 remaining events, we are able to infer the mass and spin distributions of three potential BBH subpopulations, detailed below. 

\subsection{BBH Mass and Spin Distributions}

Figures \ref{fig:mass_distribution} and \ref{fig:spin_mag_distribution} show the inferred distributions of primary mass $m_1$, spin magnitude $a$ and cosine of the tilt angle $\textsc{cos}(\theta)$ of each the three categories \first{}, \second{}, and \cont{}, as well as the total BBH population distributions of these same parameters plotted alongside the total distributions inferred by \jaxen{CITE bruce and 03a}. We see in Figure \ref{fig:mass_distribution} that the total BBH primary mass distribution inferred in this work agrees relatively well with \jaxen{CITE bruce} at the low and high mass ends of the spectrum, though the mass range covered by \second{} favors a broad Gaussian and therefore does not pick up the features inferred by \jaxen{CITE bruce} in this mass range. Since event categorization takes both mass and spin into account, this behaviour could be due to the \second{} events having similar spin properties and thus the inference algorithm may not see the need to separate these events between, say, the \second{} and \cont{} categories. 

Figure \ref{fig:mass_distribution} shows the inferred total primary mass distribution, as well as the distributions of the three categories considered, \first{}, \second{}, and \cont{}. For comparison, \ref{fig:mass_distribution} also includes the primary mass distributions inferred by \jaxen{CITE LVK GWTC-3 POP} and \jaxen{CITE BRUCE}. We find that \first{} contains a maximum at $m_\mathrm{LM,max} = $ \result{$\CIPlusMinus{\macros[Mass][LowMassPeak][max]}\msun$}  to $90\%$ credibility, which is in agreement with other recent studies that find a maximum in the BBH mass spectrum near $10 \msun$ \jaxen{CITE}. While other population analyses such as \jaxen{CITE BRUCE} find a local maximum in the mass spectrum near $35\msun$ and potentially a second local maximum near $20\msun$, the \second{} in this work contains a maximum at $m_\mathrm{HM,max} = $ \result{$\CIPlusMinus{\macros[Mass][HighMassPeak][max]}\msun$}  to $90\%$ credibility. As seen in Figure \ref{fig:mass_distribution}, the \second{} prefers a broad Gaussian primary mass distribution that washes out the features seen in \jaxen{CITE BRUCE and 03}. To understand why this occurs, it is useful to consider the spin magnitude and tilt distributions of \second{}, which we discuss below. Looking again to Figure \ref{fig:mass_distribution}, \cont{} picks up the tail end of the primary mass distribution, though from Table \ref{table:branch} and Figure \ref{fig:radar_plot} we see that at minimum only one event is confidently associated with \cont{}. 

Figure \ref{fig:spin_mag_distribution} shows the inferred spin magnitude $a$ and tilt $\textsc{cos}(\theta)$ distributions of each category, as well as the total distributions from this analysis plotted with the total distributions from \jaxen{CITE bruce and 03}. \first{} contains a peak in spin magnitude at $a_{\textsc{LM,peak}} = $\result{$\CIPlusMinus{\macros[SpinMag][LowMassPeak][max]}$} while \second{} peaks at $a_{\textsc{HM,peak}} = $\result{$\CIPlusMinus{\macros[SpinMag][HighMassPeak][max]}$} to $90\%$ credibility. The spin magnitude and tilt distributions of \cont{} are unconstrained 

\begin{figure*}[ht!]
    \begin{centering}
        \includegraphics[width=\linewidth]{figures/spin_mag_distribution_plot.pdf}
        \caption{The marginal primary spin magnitude distribution}
        \label{fig:spin_mag_distribution}
    \end{centering}
    \script{spin_distributions_plot.py}
\end{figure*}



\begin{table}[h!]
\centering
\begin{tabular} { || m{5em} | p{5em}| p{5em} | m{5.5em}||} 
\hline
& \first & \second & \cont \\
\hline \hline
Branching Ratio  & $\CIPlusMinus{\macros[BranchingRatios][LowMassPeak][Frac]}$ & $\CIPlusMinus{\macros[BranchingRatios][HighMassPeak][Frac]}$ & $\CIPlusMinus{\macros[BranchingRatios][Continuum][Frac]}$ \\ 
\hline
Number of Events  & $\macros[NumEvents][LowMassPeak][low]$ - $\macros[NumEvents][LowMassPeak][high]$ & $\macros[NumEvents][HighMassPeak][low]$ - $\macros[NumEvents][HighMassPeak][high]$ & $\macros[NumEvents][Continuum][low]$ - $\macros[NumEvents][Continuum][high]$ \\
\hline
\end{tabular}
\caption{Table to test captions and labels.}
\label{table:branch}
\end{table}

\begin{figure*}[ht!]
    \begin{centering}
        \includegraphics[width=\linewidth]{figures/radar_plot.pdf}
        \caption{}
        \label{fig:radar_plot}
    \end{centering}
    \script{radar_plot.py}
\end{figure*}