% Main body with filler text
\section{Introduction} \label{sec:intro}

The first detection of gravitational waves (GWs) from a binary black hole (BBH) merger was made by the LIGO-Virgo-KAGRA (LVK) Collaboration on September 14, 2015. Since that fateful day, the LVK has detected nearly 100 compact binary coalescences (CBCs), bringing the third gravitational wave transient catalog (GWTC-3) up to 90 such events. \citep{2015CQGra..32g4001L,2015CQGra..32b4001A,2021PTEP.2021eA102A,2016PhRvL.116f1102A,2019PhRvX...9c1040A,2021PhRvX..11b1053A,2021arXiv211103606T}. With the maturation of Graviational-Wave Astronomy, novel studies of the Universe are possible; we are now able to probe the entire population of merging compact objects in the universe with much greater fidelity than with the sparse, early LVK catalogs \citep{2019ApJ...882L..24A,2021ApJ...913L...7A,2021arXiv211103634T}. By breaking down the full CBC population into subpopulations based on different source properties- and paired with our theoretical knowledge of stellar astrophysics- we can begin to uncover the formation and evolution of compact binaries \citep{2017ApJ...846...82Z}. The two most common expected formation channels of merging compact objects are isolated formation and dynamical assembly, each predicted to produce binaries with unique spin characteristics \citep{2017Natur.548..426F,2018ApJ...854L...9F}. Recent studies have found support for a significant contribution of systems formed through dynamical assembly in the population of BBHs inferred from the GWTC-2 and GWTC-3 catalogs \citep{2021ApJ...913L...7A,2021PhRvD.104h3010R,2021arXiv211103634T,2022ApJ...937L..13C,2021ApJ...921L..15G,2022arXiv220902206T,2022arXiv220906978V,2022arXiv221012834E}, though with large uncertainties. 

The spin distribution of merging binaries is thought to provide the most direct evidence of their formation channel \citep{2017Natur.548..426F,2018ApJ...854L...9F}. Isolated binary evolution scenarios predict component spins to be preferentially aligned with the orbital angular momentum of the binary \citep{10.1051/0004-6361/201936204,10.1051/0004-6361/202039804}, while systems formed via dynamical assembly are thought to have isotropic component spin orientations \citep{10.3847/2041-8205/832/1/L2,10.1103/PhysRevD.100.043027}. With current data it is difficult to distinguish between these two channels, though studies have at least shown that GWTC-3 is not consistent with entirely dynamical or entirely isolated formation \citep{2021arXiv211103634T,2022ApJ...937L..13C,2022arXiv220902206T,2022arXiv221012834E,10.3847/2041-8213/ac86c4}. There is also evidence of additional structure in the mass distribution, with multiple peaks that could signal the exhistence of different subpopulations \citep{2021ApJ...913L..19T,2022ApJ...924..101E,2021arXiv211103634T,2022ApJ...928..155T,2022arXiv221012834E}. Due to current uncertainties in the parameter estimation of individual compact object mergers with gravitational waves, it may take many more observations to confidently disentangle competing formation channels with the spin or mass distributions alone; however, searching for correlated population properties across mass, spin, and redshift may allow us to better distinguish between the different CBC formation channels \citep{2021ApJ...912...98F,2021ApJ...922L...5C,2022ApJ...931...17V,2022ApJ...932L..19B}. 

In this letter, we search for signs of possible subpopulations in GWTC-3 by incorporating discrete latent variables in the hierarchical Bayesian inference framework to probabilistically assign each BBH observation into separate categories that are associated with distinctly different mass and spin distributions. Incorporating these discrete variables during inference allows us to easily infer posterior distributions for astrophysical branching ratios and each BBH's association with the subpopulations. The remaining sections of letter are structured as follows: Section \ref{sec:method} describes the statistical framework with the inclusion of discrete latent variables and the specific models used for separate subpopulations. Section \ref{sec:results} presents the results of our study, including the inferred branching ratios and the inferred subpopulation membership probabilities for each BBH in GWTC-3. In section \ref{sec:astro} we discuss the implications of our findings and how it relates to current understanding of compact binary formation and population synthesis. We finish in section \ref{sec:conclusion}, with a summary of the letter and prospects for distinguishing subpopulations in future catalogs after the LVK's fourth observing run. 