% Main body with filler text
\section{Introduction} \label{sec:intro}

The first detection of gravitational waves (GWs) from a binary black hole (BBH) merger was made by the LIGO-Virgo-KAGRA (LVK) Collaboration on September 14, 2015. Since that fateful day, the LVK has detected nearly 100 compact binary coalescences (CBCs), bringing the third gravitational wave transient catalog (GWTC-3) up to 90 such events. \citep{2015CQGra..32g4001L,2015CQGra..32b4001A,2021PTEP.2021eA102A,2016PhRvL.116f1102A,2019PhRvX...9c1040A,2021PhRvX..11b1053A,2021arXiv211103606T}. With the maturation of GW Astronomy, novel studies of the universe are possible; we are now able to probe the entire population of merging compact objects in the universe with much greater fidelity than with the sparse, early LVK catalogs \citep{2019ApJ...882L..24A,2021ApJ...913L...7A,2021arXiv211103634T}. By breaking down the full CBC population into subpopulations based on different source properties- and paired with our theoretical knowledge of stellar astrophysics- we can begin to uncover the formation and evolution of compact binaries \citep{2017ApJ...846...82Z}. The two most common expected formation channels of merging compact objects are isolated formation and dynamical assembly each predicted to produce binary populations with unique mass and spin characteristics \citep{2017Natur.548..426F,2018ApJ...854L...9F,10.3847/1538-4357/ab88b2}. The uncertainty in modeled merger rates of each formation channel is large, and the predictions continue to evolve with better understanding of the underlying physics (see \citet{10.1007/s41114-021-00034-3} for a thorough review on both modeled and observed merger rates of compact objects). By looking deeper at the correlations between the source properties at a population level, we can begin to look for subpopulations of observations that we can confidently associate with a specific formation channel.

Isolated formation of compact binaries occurs in galactic fields, where two gravitationally bound stars isolated from their environment undergo standard main sequence evolution, each eventually forming into a compact object. Energy loss from GW radiation causes the binary to inspiral, which can eventually lead to merger; however, in order for the binary to merge within a Hubble time, the initial orbital separation of the must be small \citep{10.1051/0004-6361/201936204,10.1007/s41114-021-00034-3}. Some process during stellar evolution is therefore required to rapidly decrease the orbital separation down to this limit for systems with much larger initial separations. One proposed mechanism is the ``common envelope phase'', in which a cloud of non-rotating gas engulfs the two objects (typically after one star has already collapsed into a compact object) and drag forces quickly dissipate orbital energy, thus reducing the orbital separation enough so that the resulting compact binary can merge due to GW emission alone \citep{10.1038/nature18322}. In dynamical formation scenarios, scattering or exchange interactions between astrophysical bodies in a dense stellar environment are thought to produce binaries capable of merging within a Hubble time \citep{1602.02444}. There are many theorized models for the main physical processes that contribute to isolated and dynamical formation, but there are few robust and direct predictions of observable quantities from these models. Instead, current predictions of merger rates and population distributions are estimated from numerical simulations, which have large uncertainties due to uncertain underlying physics or poorly constrained initial conditions \citep{10.1007/s41114-021-00034-3, 10.1051/0004-6361/201936204, 1806.00001v3, 1308.1546}. 

The spin distribution of merging binaries is thought to provide the most direct evidence of their formation channel \citep{2017Natur.548..426F,2018ApJ...854L...9F}. Isolated binary evolution scenarios predict component spins to be near zero and preferentially aligned with the orbital angular momentum of the binary, though there are processes, such as angular momentum transport and supernova kicks, that can impart a small non-zero and modestly misaligned spin to one or both of the binary objects \citep{2203.02515, 1706.07053, 10.1051/0004-6361/201936204, 10.1051/0004-6361/202039804}. On the other hand, systems assembled dynamically in stellar clusters are thought to have no preferential alignment, producing an isotropically distributed spin tilt distribution \citep{10.3847/2041-8205/832/1/L2,10.1103/PhysRevD.100.043027}. With current data it is difficult to distinguish between these two channels, though studies have at least shown that GWTC-3 is not consistent with entirely dynamical or entirely isolated formation \citep{2021arXiv211103634T,2022ApJ...937L..13C,2022arXiv220902206T,2022arXiv221012834E,10.3847/2041-8213/ac86c4}. Recent studies have found support for a significant contribution of systems formed through dynamical assembly in the population of BBHs inferred from the GWTC-2 and GWTC-3 catalogs \citep{2021ApJ...913L...7A,2021PhRvD.104h3010R,2021arXiv211103634T,2022ApJ...937L..13C,2021ApJ...921L..15G,2022arXiv220902206T,2022arXiv220906978V,2022arXiv221012834E}, though with large uncertainties. 

While spin may be the characteristic most directly linked to compact binary formation history, the LVK parameter estimation of individual event spin properties contains large uncertainties, making it difficult to disentangle competing formation channels with spin alone. However, the component masses of individual events are typically inferred with greater certainty than their spin, and there are even features in the mass distribution that may signal the existence of different subpopulations \citep{2021ApJ...913L..19T,2022ApJ...924..101E,2021arXiv211103634T,2022ApJ...928..155T,2022arXiv221012834E}. Unfortunately, it can also be challenging to distinguish between the isolated and dynamical formation channels using only component mass, as the models in both scenarios predict masses that significantly overlap \citep{1609.05916}. Instead, a search for correlated population properties across mass, spin, and redshift may prove to be much more fruitful in distinguishing between the different CBC formation channels \citep{2021ApJ...912...98F,2021ApJ...922L...5C,2022ApJ...931...17V,2022ApJ...932L..19B}. 

In this letter, we search for signs of possible BBH subpopulations in GWTC-3 by incorporating discrete latent variables in the hierarchical Bayesian inference framework to probabilistically assign each BBH observation into separate categories that are associated with distinctly different mass and spin distributions. Incorporating these discrete variables during inference allows us to easily infer each BBH's association with each category, in addition to the posterior distributions for astrophysical branching ratios. The remaining sections of letter are structured as follows: Section \ref{sec:methods} describes the statistical framework with the inclusion of discrete latent variables and the specific models used for separate subpopulations. Section \ref{sec:results} presents the results of our study, including the inferred branching ratios and the inferred subpopulation membership probabilities for each BBH in GWTC-3. In section \ref{sec:astro} we discuss the implications of our findings and how it relates to current understanding of compact binary formation and population synthesis. We finish in section \ref{sec:conclusion}, with a summary of the letter and prospects for distinguishing subpopulations in future catalogs after the LVK's fourth observing run. 