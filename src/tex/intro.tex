% Main body with filler text
\section{Introduction} \label{sec:intro}

The first detection of gravitational waves (GWs) from a binary black hole (BBH) merger was made by the LIGO-Virgo-KAGRA (LVK) Collaboration on September 14, 2015. Since that fateful day, the LVK has detected nearly 100 compact binary coalescences (CBCs), bringing the third gravitational wave transient catalog (GWTC-3) up to 90 such events. \citep{2015CQGra..32g4001L,2015CQGra..32b4001A,2021PTEP.2021eA102A,2016PhRvL.116f1102A,2019PhRvX...9c1040A,2021PhRvX..11b1053A,2021arXiv211103606T}. With the maturation of GW Astronomy, novel studies of the universe are possible; we are now able to probe the low-redshift population of merging compact objects with much greater fidelity than with the sparse, early LVK catalogs \citep{2019ApJ...882L..24A,2021ApJ...913L...7A,2021arXiv211103634T}. By breaking down the full CBC population into subpopulations based on different source properties -- and paired with our theoretical knowledge of stellar astrophysics -- we can begin to uncover the formation and evolution of compact binaries \citep{2017ApJ...846...82Z}. While there are many proposed processes that could lead to a compact binary able to merge through gravitational radiation, they largely fall within two broad categories: isolated formation and dynamical assembly.  Each of these channels could leave their own distinct imprint on the binaries they produce, including the distributions of masses and spins \citep{2017Natur.548..426F,2018ApJ...854L...9F,10.3847/1538-4357/ab88b2}. The uncertainty in modeled merger rates of each formation channel is large, and the predictions continue to evolve with better understanding of the underlying physics (see \citet{10.1007/s41114-021-00034-3} for a thorough review on both modeled and observed merger rates of compact objects). By looking deeper at the correlations between the source properties at a population level, the identification of distinct subpopulations with common mass or spin properties could help to identify unique formation mechanisms or channels.

Isolated formation of compact binaries occurs in galactic fields where two progenitors are born gravitationally bound, and isolated from interactions with other stars undergo standard main sequence evolution, each eventually forming into a compact object. Energy loss from GW radiation causes the binary to inspiral, which can eventually lead to merger; however, in order for the binary to merge within a Hubble time, the initial orbital separation of the compact objects must be small \citep{10.1051/0004-6361/201936204,10.1007/s41114-021-00034-3}. For the binary to survive the early evolutionary phases of the progenitors the orbit must initially be wide, thus some process during late-stage evolution is required to significantly reduce the orbital separation before the second compact object is formed. One proposed mechanism is the ``common envelope phase'', where both stars momentarily share an envelope (typically after one star has already collapsed into a compact object) and drag forces quickly dissipate orbital energy, thus reducing the orbital separation enough so that the resulting compact binary can merge due to GW emission alone \citep{10.1038/nature18322}. In dynamical formation scenarios, scattering and exchange interactions between astrophysical bodies in a dense stellar environment are thought to produce binaries capable of merging within a Hubble time \citep{1602.02444}. There are many theorized models for the main physical processes that contribute to isolated and dynamical formation, but there are few robust and direct predictions of observable quantities from these models. Instead, current predictions of merger rates and population distributions are estimated from simulations, which have large uncertainties due to uncertain underlying physics or poorly constrained initial conditions \citep{10.1007/s41114-021-00034-3,10.1051/0004-6361/201936204,1806.00001v3,1308.1546}.


The spin distribution of merging binaries is thought to provide the most direct evidence of their formation channel \citep{10.1088/1361-6382/aa552e, 10.1093/mnras/stx1764, 2017Natur.548..426F,2018ApJ...854L...9F}. Isolated binary evolution scenarios predict component spins to be near zero \ben{I don't think this is true; e.g., chemically homogenious evolution is an isolated formation mechanism but predicts high spins} and preferentially aligned with the orbital angular momentum of the binary, though there are processes, such as angular momentum transport and supernova kicks, that can impart modest misalignment of spins \citep{2203.02515, 1706.07053, 10.1051/0004-6361/201936204, 10.1051/0004-6361/202039804}. On the other hand, systems assembled dynamically in stellar clusters are thought to have no preferential alignment, producing an isotropically distributed spin tilt distribution \citep{1609.05916,10.1103/PhysRevD.100.043027}. With current data it is difficult to distinguish between these two channels, though studies have at least shown that GWTC-3 is not consistent with entirely dynamical or entirely isolated formation \citep{2021arXiv211103634T,2022ApJ...937L..13C,2022arXiv220902206T,2022arXiv221012834E,10.3847/2041-8213/ac86c4,2011.09570}. Recent studies have found support for a significant contribution of systems formed through dynamical assembly in the population of BBHs inferred from the GWTC-2 and GWTC-3 catalogs \citep{2021ApJ...913L...7A,2021PhRvD.104h3010R,2021arXiv211103634T,2022ApJ...937L..13C,2021ApJ...921L..15G,2022arXiv220902206T,2022arXiv220906978V,2022arXiv221012834E}, though with large uncertainties.

While spin may be the characteristic most directly linked to compact binary formation history, spin measurements of individual binaries typically have large uncertainties, making it difficult to disentangle competing formation channels with spin alone. However, the component masses of individual events are typically inferred with higher precision than their spins, and there are even features in the mass distribution that may signal the existence of different subpopulations \citep{2021ApJ...913L..19T,2022ApJ...924..101E,2021arXiv211103634T,2022ApJ...928..155T,2022arXiv221012834E}. Unfortunately, it can also be challenging to distinguish between the isolated and dynamical formation channels using only component mass, as the models in both scenarios predict masses that significantly overlap \citep{1609.05916}. Instead, a search for correlated population properties across mass, spin, and redshift may prove to be much more fruitful in distinguishing between the different compact binary formation channels \citep{2021ApJ...912...98F,2021ApJ...922L...5C,2022ApJ...931...17V,2022ApJ...932L..19B}.

Given the evidence for features, e.g. peaks, in the primary mass spectrum (see Figure \ref{fig:g1_mass_distribution} and \citet{2021arXiv211103634T, 2022ApJ...928..155T, 10.3847/2041-8213/aa9bf6, 10.3847/1538-4357/aab34c, 2019ApJ...882L..24A, 2021ApJ...913L...7A}), in this letter we attempt to identify subpopulations of BBHs grouped in primary mass with spin properties distinct from the rest of the GWTC-3 BBH population. We do this by modeling a portion of the BBH primary mass spectrum as a log-normal peak and allowing the spin distributions of the events categorized in this peak to differ from the rest of the population. As we will detail below, we identified such a subpopulation and then conducted a follow-up search for events with different primary masses but similar spin properties to these events.

To distinguish subpopulations, we incorporate discrete latent variables in the hierarchical Bayesian inference framework to probabilistically assign events to categories with unique mass and spin distributions. Incorporating these discrete variables during inference allows us to easily infer each BBH's association with each category, in addition to the posterior distributions for astrophysical branching ratios (also referred to as mixing fractions). \ben{Previous applications of mixture models to the binary catalog (CITE) implicitly marginalize over event categories.  The population properties inferred using these approaches are identical, and event categorization probabilities can be calculated, but sampling these latent variables enables detailed investigation of correlations between categorization of events and population properties.}

The remaining sections of this letter are structured as follows: Section \ref{sec:methods} describes the statistical framework with the inclusion of discrete latent variables and the different component models studied. Section \ref{sec:results} presents the results of our study, including the inferred branching ratios and the inferred subpopulation membership probabilities for each BBH in GWTC-3. In section \ref{sec:astro} we discuss the implications of our findings and how it relates to the current understanding of compact binary formation and population synthesis. We finish in section \ref{sec:conclusion}, with a summary of the letter and prospects for distinguishing subpopulations in future catalogs after the LVK's fourth observing run.