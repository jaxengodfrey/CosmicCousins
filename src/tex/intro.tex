% Main body with filler text
\section{Introduction} \label{sec:intro}

Gravitational-Wave (GW) Astronomy is beginning to mature, enabling novel studies of the Universe. The first detection of GWs from a binary black hole (BBH) merger was made by the LIGO-Virgo-KAGRA (LVK) Collaboration in 2015, which now is producing catalogs, with the recent third gravitational-wave transient catalog (GWTC-3) 90 \citep{2015CQGra..32g4001L,2015CQGra..32b4001A,2021PTEP.2021eA102A,2016PhRvL.116f1102A,2019PhRvX...9c1040A,2021PhRvX..11b1053A,2021arXiv211103606T}. With the ever-growing catalog sizes, we can now probe the full population of merging compact objects in the universe with greater fidelity \citep{2019ApJ...882L..24A,2021ApJ...913L...7A,2021arXiv211103634T}. By characterizing this population, we start to gain a deeper understanding of the formation and evolution of compact binaries. The two most common expected formation channels of merging compact objects are isolated formation and dynamical assembly, each predicted to produce binaries with unique spin characteristics. Recent studies have found support for a significant contribution of systems formed through dynamical assembly in the population of BBHs inferred from the GWTC-2 and GWTC-3 catalogs \citep{2021ApJ...913L...7A,2021PhRvD.104h3010R,2021arXiv211103634T,2022ApJ...937L..13C,2021ApJ...921L..15G,2022arXiv220902206T,2022arXiv221012834E}, though with large uncertainties. 

The spin distribution of merging binaries is thought to provide the most direct evidence for the formation channel of the binary, with the isolated binary evolution scenarios predicting spins that are preferentially aligned with the orbital angular momentum of the binary, while dynamical assembly should have isotropic orientations. With current estimates it is difficult to distinguish between these two channels, but studies have shown that GWTC-3 is not consistent with entirely dynamical formation or entirely isolated. There was also evidence of additional structure in the mass distribution, with multiple peaks that also could be the sign of multiple subpopulations. With current uncertainties in parameter estimation of compact binaries with gravitational waves, it may take many more observations to confidently disentangle competing formation channels through the spin or mass distributions alone. By looking for correlated population properties across mass, spin and redshift, we better distinguish between different formation channels of compact binary mergers. 

In this letter, we search for signs of possible such subpopulations in GWTC-3 by incorporating discrete latent variables in the hierarchical Bayesian inference framework that probabilistically assign each BBH observation into separate categories that are associated with distinctly different population distributions. Incorporating these discrete variables during inference allows us to easily infer posterior distributions for each BBH's association with the subpopulations and astrophysical branching ratios. The remaining sections of letter are structured as follows: Section \ref{sec:method} describes the statistical framework with the inclusion of discrete latent variables and the specific models used for separate subpopulations. Section \ref{sec:results} presents the results of our study, including the inferred branching ratios and the inferred subpopulation membership probabilities for each BBH in GWTC-3. In section \ref{sec:astro} we discuss the implications of our findings and how it relates to current understanding of compact binary formation and population synthesis. We finish in section \ref{sec:conclusion}, with a summary of the letter and prospects for distinguishing subpopulations in future catalogs after the LVK's fourth observing run. 